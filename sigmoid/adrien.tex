%
%  untitled
%
%  Created by Johan Boissard [] on 2010-06-24.
%  Copyright (c) Johan Boissard. All rights reserved.
% hhh

\documentclass[a4paper,titlepage] {scrartcl}
\usepackage[T1]{fontenc}
\usepackage[utf8]{inputenc}
\usepackage{graphicx}
\usepackage{engord}
%\usepackage[english]{babel}
\usepackage{fancyhdr}
\usepackage{amsmath}
\usepackage{comment}

\usepackage{listings}

%allows inclusion of url (hyperref is better than url) 
%ref: http://www.fauskes.net/nb/latextips/
\usepackage{hyperref}

%package for chemistry ie: \ce{(NH4)2SO4 -> NH4+ + 2SO4^2-} 
%ref:www.ctan.org/tex-archive/macros/latex/contrib/mhchem/mhchem.pdf
\usepackage[version=3]{mhchem}
%celsius + degrees
\usepackage{gensymb}
%to get last page
\usepackage{lastpage} % \pageref{LastPage}

%make use of the fullpage (no HUGE margins)
\usepackage{fullpage}
\usepackage{subfig}

%allows separating cell in table by diagonal line
\usepackage{slashbox}




%\renewcommand{\chaptername}{Laboratory}
%\setcounter{chapter}{5}

\usepackage{color}
\usepackage[usenames,dvipsnames, table]{xcolor}
% Include this somewhere in your document



\usepackage[absolute]{textpos}

%column  of multi row in tables
\usepackage{multirow}

%to have vertical text in table
\usepackage{rotating}


%%%%%%% a virer ici!!!!
\begin{comment}
%Fonts and Tweaks for XeLaTeX
\usepackage{fontspec,xltxtra,xunicode}
%\defaultfontfeatures{Mapping=tex-text}
%\setromanfont[Mapping=tex-text]{Hoefler Text}
\setsansfont[Scale=MatchLowercase,Mapping=tex-text]{Gill Sans}

\definecolor{shade}{HTML}{D4D7FE}	%light blue shade
\definecolor{text1}{HTML}{272727}		%text is almost black
\definecolor{headings}{HTML}{173849} 	%dark blue %%%dark red 70111
\definecolor{title}{HTML}{173849} 	%dark blue %%%dark red 70111

\usepackage{titlesec}				%custom \section
\end{comment}







\author{Johan Boissard}
\date{\today}
\title{Regression S-curve}
\begin {document}

%\maketitle
%\tableofcontents

\section{Théorie}
Je sais pas ce que tu entends par sigmoide (il y a plusieurs types: logit, $tanh$, $\frac{x}{\sqrt{1+x^2}}$, ...), je te propose ici une méthode pour trouver les paramètres de la S-curve (sigmoide) pour 
\begin{equation}
	f(t)= \frac{K}{1+e^{-a-bt}}=\frac{K}{1+Ce^{-bt}}
\end{equation}
où $K$ est la valeur de saturation ($\lim_{t\rightarrow\infty}f(t)=K$). $a$ et $b$, deux paramètres que tu vas déterminer et $C=e^{-a}$.

Cette dernière me paraît la plus naturelle, car elle est la résolution de l'équation différentielle suivante
\begin{equation}
	\dot f = k(N-f)f
\end{equation}

En prenant le log des deux côtés et en reformant un peu, tu obtiens:
\begin{equation}
	z=\ln{\frac{f}{K-f}}=a+bt
\end{equation}
Qui est cette fois une fonction linéraire. Pour trouver $a$ et $b$, il te suffit de faire une imple régression linéaire sur excel par exemple.

\section{Pratique (excel)}
\begin{enumerate}
	\item créer deux colonnes, $f$ (ordonnée) et $t$ (abscisse).
	\item déterminer graphiquement la valeur de saturation (faire le graph et voir vers quoi la fonction tend, $K$)
	\item créer une nouvelle colonne avec comme valeur $z=\ln{\frac{f}{K-f}}$, où $K$ est la valeur vers laquelle la fonction tend.
	\item faire une régression linéaire pour $z=a+bt$
\end{enumerate}


Voilà, j'espère que ca réout ton problème, ++
\end{document}
