%
%  untitled
%
%  Created by Johan Boissard [] on 2010-06-24.
%  Copyright (c) Johan Boissard. All rights reserved.
% hhh

\documentclass[a4paper] {scrartcl}%add titlepage for title separate
\usepackage[T1]{fontenc}
\usepackage[utf8]{inputenc}
\usepackage{graphicx}
\usepackage{engord}
%\usepackage[english]{babel}
\usepackage{fancyhdr}
\usepackage{amsmath, amssymb}
\usepackage{comment}

\usepackage{listings}

%allows inclusion of url (hyperref is better than url) 
%ref: http://www.fauskes.net/nb/latextips/
\usepackage{hyperref}

%package for chemistry ie: \ce{(NH4)2SO4 -> NH4+ + 2SO4^2-} 
%ref:www.ctan.org/tex-archive/macros/latex/contrib/mhchem/mhchem.pdf
\usepackage[version=3]{mhchem}
%celsius + degrees
\usepackage{gensymb}
%to get last page
\usepackage{lastpage} % \pageref{LastPage}

%make use of the fullpage (no HUGE margins)
\usepackage{fullpage}
\usepackage{subfig}

%allows separating cell in table by diagonal line
\usepackage{slashbox}




%\renewcommand{\chaptername}{Laboratory}
%\setcounter{chapter}{5}

\usepackage{color}
\usepackage[usenames,dvipsnames, table]{xcolor}
% Include this somewhere in your document



\usepackage[absolute]{textpos}

%column  of multi row in tables
\usepackage{multirow}

%to have vertical text in table
\usepackage{rotating}


%%%%%%% a virer ici!!!!
\begin{comment}
%Fonts and Tweaks for XeLaTeX
\usepackage{fontspec,xltxtra,xunicode}
%\defaultfontfeatures{Mapping=tex-text}
%\setromanfont[Mapping=tex-text]{Hoefler Text}
\setsansfont[Scale=MatchLowercase,Mapping=tex-text]{Gill Sans}

\definecolor{shade}{HTML}{D4D7FE}	%light blue shade
\definecolor{text1}{HTML}{272727}		%text is almost black
\definecolor{headings}{HTML}{173849} 	%dark blue %%%dark red 70111
\definecolor{title}{HTML}{173849} 	%dark blue %%%dark red 70111

\usepackage{titlesec}				%custom \section
\end{comment}







\author{Johan Boissard}
\date{\today}
\title{Derivative}
\begin {document}

\maketitle
%\tableofcontents

\section{What 's a derivative}

We define a function $f(x): \mathbb R \rightarrow\mathbb R$, taking $x$ as argument; for every $x$ there is one (and only one) value $f(x)$ associated. $f'(x)$ denotes the \textbf{derivative} of $f(x)$.

\subsection{Intuitive definitions}
Basically, the derivatives says \textbf{how steep is $f(x)$ at $x$}. 
\\\\Or a little bit more mathematically:



\begin{eqnarray}
	f'(x)=\frac{\Delta f(x)}{\Delta x}
\end{eqnarray}
$\dots$ but when $\Delta x$ is very small.

\subsection{Three cases}
\begin{itemize}
	\item $f'(x)>0$ "at $x$ "goes up"
	\item $f'(x)<0$ at $x$ "goes down"
	\item $f'(x)=0$ at $x$ "remains stationary" $\rightarrow$ local optimum
\end{itemize}

See \url{http://en.wikipedia.org/wiki/Derivative} for illustration.

\section{Very small $\rightarrow$ Infinitesimal}
\begin{eqnarray}
	\frac{\Delta f(x)}{\Delta x}
\end{eqnarray}

\begin{eqnarray}
	\lim_{x\rightarrow0}\Delta x=dx
\end{eqnarray}

\subsection{Official Definition}
\begin{eqnarray}
	f'(x)=\frac{df}{dx}=\lim_{h\rightarrow0}\frac{f(x+h)-f(x)}{h}
\end{eqnarray}

\subsubsection{Notations}
Note that $\frac{df}{dx}$ (Leibniz's notation) is strictly equal to $f'(x)$ (Lagrange notation). However the notation $\frac{df}{dx}$ should be preferred since it really represents what differentiation is: the measure of how a function change ($df$) as its input changes ($dx$).

$\frac{d}{dx}$ is the operator for differentiation, thus one can simply write
\begin{eqnarray*}
	\frac{d(7x^3)}{dx}
\end{eqnarray*} 
instead of
\begin{eqnarray*}
	f'(x), &\text{where } f(x)=7x^3
\end{eqnarray*} 

Note that some people (depending on the context) write it like this: $\dot f$ (Newton's notation) or $D_x f$ (Euler's notation)

\subsection{Some examples}
for $f(x)=ax$
\begin{eqnarray*}
	\frac{df}{dx}
	&=&\lim_{h\rightarrow0}\frac{a(x+h)-ax}{x+h-x}\\
	&=&\lim_{h\rightarrow0}\frac{ah}{h}\\
	&=&\lim_{h\rightarrow0}ah\\
	&=&a
\end{eqnarray*}

So for a linear function, $f(x)=ax+b$ (the previous example can be easily generalized), the derivative is a \textbf{constant}. This is only valid for linear function.

for $f(x)=ax^2$

\begin{eqnarray*}
	\frac{df}{dx}
	&=&\lim_{h\rightarrow0}\frac{a(x+h)^2-ax^2}{x+h-x}\\
	&=&\lim_{h\rightarrow0}a\frac{x^2+2hx+h^2-x^2}{h}\\
	&=&\lim_{h\rightarrow0}a\frac{2hx+h^2}{h}\\
	&=&\lim_{h\rightarrow0}a(2x+h)\\
	&=&2ax
\end{eqnarray*}

for $f(x)=e^{x}$

\begin{eqnarray*}
	\frac{df}{dx}
	&=&\lim_{h\rightarrow0}\frac{e^{x+h}-e^{h}}{x+h-x}\\
	&=&\lim_{h\rightarrow0}\frac{e^x(e^h-1)}{h}\\
	&=&e^x\underbrace{\lim_{h\rightarrow0}\frac{e^h-1}{h}}_{1}\\
	&=&e^x
\end{eqnarray*}

\subsection{Relation between angle and slope}
One can intuitively notice that there is a tight relation between "slope" and angle, in fact this relation is very precise and is the following:
\begin{eqnarray}
	\theta = \arctan{(f'(x))}
\end{eqnarray}
where $\theta$ is the angle (in radians) between the function $f$ and the horizontal axis at point $x$. Once again, if $f'(x)=c$ the angle is also constant for very $x$ and this is only the case for linear functions. 

If $f'(x)=1$, then $\theta=\frac{\pi}{4}=45\degree$, this is the case when the variation of $f$ is the same as the variation of $x$.

If $f'(x)=0$, then $\theta=0$: there is no variation in $f$ ($df=0$).

If $f'(x)=\infty$, then $\theta=0$: there is a huge variation in $f$, the curve is vertical at $x$ this is a limitation of the derivative (there are methods to overcome this problem, but they are beyond the scope of that tutorial).


\section{Rules for calculating derivatives}
Using limits is time consuming and not very practical, fortunately there exist some properties that allows to calculate derivatives in a much simpler way.

\subsection{Properties}
\begin{itemize}
	\item Linearity
	\begin{eqnarray*}
		f(x)=ag(x)\\
		f'(x)=ag'(x)
	\end{eqnarray*}
	\begin{eqnarray*}
		f(x)=g(x)+h(x)\\
		f'(x)=g'(x)+h'(x)
	\end{eqnarray*}
	\item Constants disappear
	\begin{eqnarray*}
		f(x)=1\\
		f'(x)=0
	\end{eqnarray*}
	\begin{eqnarray*}
		f(x)=a\\
		f'(x)=0
	\end{eqnarray*}
	
	\item Polynomes
	\begin{eqnarray*}
		f(x)=x^n\\
		f'(x)=nx^{n-1}
	\end{eqnarray*}
	
	\item Products
	\begin{eqnarray*}
		f(x)=g(x)\cdot h(x)\\
		f'(x)=g'(x)\cdot h(x)+g(x)\cdot h'(x)
	\end{eqnarray*}
	
	\item Division
	\begin{eqnarray*}
		f(x)=\frac{g(x)}{h(x)}\\
		f'(x)=\frac{g'(x)\cdot h(x)-g(x)\cdot h'(x)}{h^2(x)}
	\end{eqnarray*}
	
	\begin{eqnarray*}
		f(x)=\frac{1}{g(x)}\\
		f'(x)=-\frac{g'(x)}{g^2(x)}
	\end{eqnarray*}
	
	\item Exponentials
	\begin{eqnarray*}
		f(x)=f'(x)=e^{x}
	\end{eqnarray*}
	
	\begin{eqnarray*}
		f(x)=e^{h(x)}\\
		f'(x)=h'(x)e^{h(x)}
	\end{eqnarray*}
	
	\item Logarithms
	\begin{eqnarray*}
		f(x)=\ln|x|\\
		f'(x)=\frac{1}{x}
	\end{eqnarray*}
	
	\begin{eqnarray*}
		f(x)=\ln|g(x)|\\
		f'(x)=\frac{g'(x)}{g(x)}
	\end{eqnarray*}
\end{itemize}

\subsection{Examples}

\section{$2^{\text{nd}}$ order derivatives}
Until now, we only discussed about $1^{\text{st}}$ order derivatives.

$2^{\text{nd}}$ order derivatives are defined as follows	
\begin{eqnarray}
	f''(x)=\frac{d^2 f}{dx^2} = \frac{d}{dx}\left(\frac{df}{dx}\right)
\end{eqnarray}

In order to find, the $2^{\text{nd}}$ order derivative of a function $f(x)$, one just has to take \textbf{the derivative of the derivative}.

\subsection{Examples}
for $f(x)=x^2$, we have
\begin{eqnarray*}
	f'(x)&=&\frac{df}{dx}=2x\\
	f''(x)&=& \frac{d}{dx}\left(\frac{df}{dx}\right)\\
	f''(x)&=& \frac{d}{dx}\left(f'(x)\right)\\
	f''(x)&=& \frac{d}{dx}\left(2x\right)\\
	f''(x)&=& 2	
\end{eqnarray*} 

for $f(x)=e^{-x}$ we have

\begin{eqnarray*}
	f''(x)&=& \frac{d}{dx}\left(\frac{df}{dx}\right)\\
	f''(x)&=& \frac{d}{dx}\left(\frac{d(e^{-x})}{dx}\right)\\
	f''(x)&=& \frac{d}{dx}\left(f'(x)\right)\\
	f''(x)&=& \frac{d}{dx}\left(-e^{-x}\right)\\
	f''(x)&=& e^{x}
\end{eqnarray*}

\section{Practical Uses of the Derivatives}
\subsection{Optimum}
In order to find an optimum of $f$, the optimum $x^*$ has to satisfy the following relation
	\begin{eqnarray}
		f(x^*)=0
	\end{eqnarray}
	
There are 3 kinds of optimum
\begin{itemize}
	\item local maximum
	\item local minimum
	\item saddle point
\end{itemize}

To differentiate those different category (it can be a good thing to be able to tell the difference between the $x^*$ that \underline{maximizes} the cost and the $x^*$ that \underline{minimizes} the cost...), we have the following relation.

$x^*$ must satisfy $f'(x^*)=0$ \underline{and}
\begin{itemize}
	\item local maximum $\Leftarrow f''(x)<0$ 
	\item local minimum $\Leftarrow f''(x)>0$ 
	\item saddle point $\Leftarrow f''(x)=0$ 
\end{itemize}

\section{Partial Derivative}
Suppose a function of more than one variable, $f(x_1, x_2, \dots, x_n)$.

The partial derivative of $f$ with respect to $x_k$ is
\begin{eqnarray}
	\frac{\partial f}{\partial x_k}.
\end{eqnarray}

\subsection{Example}
If we set
\begin{eqnarray*}
	f(x,y) = x^2+xy+y^2
\end{eqnarray*}
we have the following partial derivatives
\begin{eqnarray*}
	\frac{\partial f}{\partial x}&=&2x+y\\
	\frac{\partial f}{\partial y}&=&x+2y
\end{eqnarray*}

\section{Total derivative}
\begin{eqnarray}
	\frac{df}{dt}=
	\frac{\partial f}{\partial t}
	+\frac{\partial f}{\partial x}\frac{dx}{dt}
	+\frac{\partial f}{\partial y}\frac{dy}{dt}
	+\frac{\partial f}{\partial z}\frac{dz}{dt}
\end{eqnarray}
or
\begin{eqnarray}
	df=
	\frac{\partial f}{\partial t}
	+\frac{\partial f}{\partial x}dx
	+\frac{\partial f}{\partial y}dy
	+\frac{\partial f}{\partial z}dz
\end{eqnarray}

\subsection{Differential operator}
\begin{eqnarray}
	\frac{d}{dx}=
	\frac{\partial f}{\partial x}
	+\sum_{i=1}^k\frac{dy_i}{dx}\frac{\partial }{\partial y_i}
\end{eqnarray}

\end{document}
