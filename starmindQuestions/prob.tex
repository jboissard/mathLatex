%
%  untitled
%
%  Created by Johan Boissard [] on 2009-07-27.
%  Copyright (c) Johan Boissard. All rights reserved.
% hhh
\documentclass[a4paper]{article}

% Packages
\usepackage[french]{babel}
\usepackage[utf8]{inputenc}
\usepackage[T1]{fontenc}
\usepackage{amsmath,amssymb, tikz, fullpage, fancyhdr}
\DeclareGraphicsExtensions{.pdf, .jpg, .tif}

% Title
\title{}
\author{Johan Boissard}
\date\today

% Header
\setlength{\headheight}{15.2pt}
\setlength{\headsep}{15pt}
\pagestyle{fancy}

\fancyhf{}
\lhead{}
\chead{}
\rhead{Johan Boissard}

\begin{document}
%\maketitle
\section{Problem}

\begin{eqnarray}
	i^i
\end{eqnarray}
where $i=\sqrt{-1}$.

\section{Solution}
First we rewrite the term using the exponential (we recall that $a^b=e^{b\ln{a}}$)
\begin{eqnarray}
	\label{eq5}i^i=e^{i\ln{(i)}}.
\end{eqnarray}
The logarithm of a complex number number can also be written
\begin{eqnarray}
	\label{eq1}\ln{(z)}=\ln{|z|}+i(\mathrm{arg}(z)+2\pi k)&k\in\mathbb Z
\end{eqnarray}
were - if $z=x+iy$ - $|z|=\sqrt{x^2+y^2}$ and $\mathrm{arg}(z)=\arctan(\frac{y}{x})$. In our case $z=i=iy \Rightarrow y=\mathrm{Im}(z)=1, x=\mathrm{Re}(z)=0$.

Thus we have
\begin{eqnarray}
	\label{eq2}|z|=\sqrt{x^2+y^2}=\sqrt{0+1^2}=1
	\\
	\label{eq3}\mathrm{arg}(z)=\arctan(\frac{y}{x})=\lim_{u\rightarrow\infty}\arctan(u)=\frac{\pi}{2}
\end{eqnarray}
We substitute equations \ref{eq2} and \ref{eq3} into equation \ref{eq1} which leads to
\begin{eqnarray}
	\label{eq8}\ln{(z)}=\ln{|z|}+i(\mathrm{arg}(z)+2\pi k)
	\\
	\label{eq4}=0+i(\frac{\pi}{2}+2\pi k)=i\pi(\frac{1}{2}+2k)
\end{eqnarray}
Finally, we substitute equation \ref{eq4} into equation \ref{eq5} (we remember that $i\cdot i=-1$) and we have
\begin{eqnarray}
	i^i=e^{i\ln{(i)}}=e^{-\pi(\frac{1}{2}+2k)}&k\in\mathbb Z
\end{eqnarray}
For $k=0$, 
\begin{eqnarray}
	i^i=e^{-\frac{\pi}{2}}\approx 0.2079
\end{eqnarray}
\end{document}
