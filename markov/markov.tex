% !TEX TS-program = pdflatex
% !TEX encoding = UTF-8 Unicode

% This is a simple template for a LaTeX document using the "article" class.
% See "book", "report", "letter" for other types of document.

\documentclass[11pt]{article} % use larger type; default would be 10pt

\usepackage[utf8]{inputenc} % set input encoding (not needed with XeLaTeX)


\usepackage{amsmath}

%%% Examples of Article customizations
% These packages are optional, depending whether you want the features they provide.
% See the LaTeX Companion or other references for full information.

%%% PAGE DIMENSIONS
\usepackage{geometry} % to change the page dimensions
\geometry{a4paper} % or letterpaper (US) or a5paper or....
% \geometry{margin=2in} % for example, change the margins to 2 inches all round
% \geometry{landscape} % set up the page for landscape
%   read geometry.pdf for detailed page layout information

\usepackage{graphicx} % support the \includegraphics command and options

% \usepackage[parfill]{parskip} % Activate to begin paragraphs with an empty line rather than an indent

%%% PACKAGES
\usepackage{booktabs} % for much better looking tables
\usepackage{array} % for better arrays (eg matrices) in maths
\usepackage{paralist} % very flexible & customisable lists (eg. enumerate/itemize, etc.)
\usepackage{verbatim} % adds environment for commenting out blocks of text & for better verbatim
\usepackage{subfig} % make it possible to include more than one captioned figure/table in a single float
% These packages are all incorporated in the memoir class to one degree or another...

\usepackage{tikz}

%%% HEADERS & FOOTERS
\usepackage{fancyhdr} % This should be set AFTER setting up the page geometry
\pagestyle{fancy} % options: empty , plain , fancy
\renewcommand{\headrulewidth}{0pt} % customise the layout...
\lhead{}\chead{}\rhead{}
\lfoot{}\cfoot{\thepage}\rfoot{}

%%% SECTION TITLE APPEARANCE
\usepackage{sectsty}
\allsectionsfont{\sffamily\mdseries\upshape} % (See the fntguide.pdf for font help)
% (This matches ConTeXt defaults)

%%% ToC (table of contents) APPEARANCE
\usepackage[nottoc,notlof,notlot]{tocbibind} % Put the bibliography in the ToC
\usepackage[titles,subfigure]{tocloft} % Alter the style of the Table of Contents
\renewcommand{\cftsecfont}{\rmfamily\mdseries\upshape}
\renewcommand{\cftsecpagefont}{\rmfamily\mdseries\upshape} % No bold!

%%% END Article customizations

%%% The "real" document content comes below...

\title{Brief Article}
\author{The Author}
%\date{} % Activate to display a given date or no date (if empty),
         % otherwise the current date is printed 

\begin{document}
\maketitle
\section{Markov Chain}

A markov chain describes the evolution of a system state over time given that we know the change of state probability for each time and state. It is described by 

\begin{equation}
\mathbf{b}_{n+1} = A\mathbf{b}_{n}
\end{equation}

where $\mathbf{b}$ is a $n\times 1$ vector and $A$ a $n\times n$ matrix; $n$ denoting the number of state in the system. The element $a_{ij}$ can be read as the probability to jump from state $i$ to state $j$. Note that $\sum_{j=1}^n a_{ij}=1~\forall i\in(1, .., n)$, $\sum_{i=1}^nb_i=1$ and $a_{ij},b_i\in(0,1)$ 

If $A$ remains constant over time ($A=A(n)$) the state of the system at time $n$ is
\begin{equation}
\mathbf{b}_n = A^{n-k}\mathbf{b}_k= A^n\mathbf{b}_0
\end{equation}

Sometimes there exists a state where the system reaches an equilibrium and is described when

\begin{equation}
\mathbf{b} = A\mathbf{b}
\end{equation}

\subsection{Market evolution}

The market shift from bull to bear and recession can be described (see wikipedia example and PUT IMAGE) using a Markov chain.

If state 
\begin{enumerate}
\item is Bull market
\item is Bear Market
\item is recession
\end{enumerate}
and we have

\begin{equation}
A =
\begin{pmatrix}
0.9	& 0.15	& .25 \\
0.075	& .8		& .25\\
0.025 & 0.05		& .5
\end{pmatrix}
\end{equation}

(e.g. transition from bull to bear market is $a_{21}=7.5\%$). One can show that eventually the market will tend to
\begin{equation}
\mathbf{b} = A\mathbf{b} = 
\begin{pmatrix}
62.5\%\\
31.25\%\\
6.25\%
\end{pmatrix}
\end{equation}



\end{document}
