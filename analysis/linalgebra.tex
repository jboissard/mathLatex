\chapter{Grands concepts}
\begin{myDefinition}
	Matrices $m\times n$ où $m$ est le nombre de lignes et $n$ le nombre de colonnes.
\end{myDefinition}
\begin{eqnarray*}
	\begin{cases}
		4x_1+6x_2+9x_3=6\\
		6x_1+2x_3=20\\
		5x_1-8x_2+x_3=10
	\end{cases}
\end{eqnarray*}
Le système ci-dessus peut être récrit sous la forme suivante
\begin{eqnarray*}
	\begin{bmatrix}
		4 &6& 9\\
		6  &0 &-2\\
		5 & -8 & 1
	\end{bmatrix}
	\begin{bmatrix}
		x_1\\x_2\\x_3
	\end{bmatrix}
	=\begin{bmatrix}
		6\\20\\10
	\end{bmatrix}
	%\mathbf{Ax}=\mathbf{B}
\end{eqnarray*}


\section{Matrix addition and scalar multiplication}
\begin{myDefinition}
	Sum of two matrices $A[a_{jk}]$ and $B[b_{jk}]$ of the same size.
	\begin{eqnarray}
		\textbf{A+B}=
		\begin{bmatrix}
			a_{11}+b_{11} & a_{12}+b_{12} & \cdots & a_{1n}+b_{1n}
			\\
			\vdots & & & \vdots
			\\
			a_{m1}+b_{m1} & a_{m2}+b_{m2} & \cdots & a_{mn}+b_{mn}
		\end{bmatrix}
	\end{eqnarray}
\end{myDefinition}
\section{Multiplying by a scalar}
\begin{myDefinition}
	\begin{eqnarray}
		c\textbf{A}=
		\begin{bmatrix}
			ca_{11}&ca_{12}&\cdots&ca_{1n}
			\\
			\vdots&&&\vdots
			\\
			ca_{m1}&ca_{m2}&\cdots&ca_{mn}
		\end{bmatrix}	
	\end{eqnarray}
	
\end{myDefinition}
\section{Matrix multiplication}
\begin{eqnarray}
	\mathbf{A}\cdot \mathbf{B}=\mathbf{C} 
\end{eqnarray}

\begin{eqnarray}
	\mathbf{A}\cdot \mathbf{B}\neq\mathbf{B}\cdot \mathbf{A}
\end{eqnarray}
[n x m][m x p]=[n x p]
\begin{eqnarray}
	c_{jk}=\sum_{i=1}^na_{ji}b_{ik}
\end{eqnarray}
\begin{myExample}
	Exemple de multiplication d'une matrice $2\times2$ par une matrice $2\times1$.
	\begin{eqnarray*}
		\begin{pmatrix}
			4 & 2 \\ 1 & 8
		\end{pmatrix}
		\cdot
		\begin{pmatrix}
			3\\5
		\end{pmatrix}
		=
		\begin{pmatrix}
			4\cdot3+2\cdot5 \\ 1\cdot3+8\cdot5
		\end{pmatrix}
		=\begin{pmatrix}
			22\\43
		\end{pmatrix}.
	\end{eqnarray*}
	Exemple de multiplication d'une matrice $2\times3$ par une matrice $3\times3$.
	\begin{eqnarray*}
		\begin{pmatrix}
			1 & 2 & 3
			\\
			4 & 5 & 6
		\end{pmatrix}
		\cdot
		\begin{pmatrix}
			3 & 2 & 0
			\\
			1 & 0 & 6
			\\
			1 & 1 & 0
		\end{pmatrix}
		=\begin{pmatrix}
			1\cdot3+2\cdot1+3\cdot1 & 1\cdot2+2\cdot0+3\cdot1 & 1\cdot0+2\cdot6+3\cdot0
			\\
			4\cdot3+5\cdot1+6\cdot1 & 4\cdot2+5\cdot0+6\cdot1 & 4\cdot0+5\cdot6+6\cdot0
		\end{pmatrix}
		=\begin{pmatrix}
			8 & 5 & 12
			\\
			23 & 14 & 30
		\end{pmatrix}
	\end{eqnarray*}
\end{myExample}

\section{Transposition}
\begin{myDefinition}
	si $\mathbf{A}[a_{jk}]$, alors $\mathbf{A^T}[a_{kj}]$.
	\begin{eqnarray}
		\mathbf{A^T}=[a_{kj}]=
		\begin{pmatrix}
			a_{11} & a_{21} & \cdots & a_{n1}
			\\
			\vdots & & &\vdots
			\\
			a_{1m} & a_{2m} &\cdots & a_{nm}
		\end{pmatrix}
	\end{eqnarray}
\end{myDefinition}
\begin{myExample}
	\begin{eqnarray*}
		\mathbf{A}=
		\begin{pmatrix}
			5 & -8 & 1
			\\
			4 & 0 & 7
		\end{pmatrix}
	\end{eqnarray*}
	Alors
	\begin{eqnarray*}
		\mathbf{A^T}=
		\begin{pmatrix}
			5 & 4\\
			-8&0\\
			1& 7
		\end{pmatrix}
	\end{eqnarray*}
\end{myExample}

\subsection{Propriétés}
\begin{eqnarray}
	\mathbf{(A^T)^T}=\mathbf{A}
	\\
	\mathbf{A^T}+\mathbf{B^T}=\mathbf{(A+B)^T}\\
	\mathbf{A^TB^T}=\mathbf{(AB)^T}
\end{eqnarray}

\begin{myDefinition}
	Symmetric matrix
	\begin{eqnarray}
		\mathbf{A}=\mathbf{A^T}
	\end{eqnarray}
	Skew symmetric matrix
	\begin{eqnarray}
		\mathbf{A^T}=-\mathbf{A}
	\end{eqnarray}
	thus $a_{jk}=-a_{kj}$ and $a_{jj}=0$.
\end{myDefinition}

\begin{myDefinition}
	Triangular matrix
	\begin{itemize}
		\item upper triangular
		\begin{eqnarray}
			a_{jk}=0&j>k
		\end{eqnarray}
		\underline{Example}
		\begin{eqnarray*}
			\begin{pmatrix}
				1 & 4 & 3\\
				0 & 5 & -8\\
				0 & 0 & 7
			\end{pmatrix}
		\end{eqnarray*}
		\item lower triangular
		\begin{eqnarray}
			a_{jk}=0&j<k
		\end{eqnarray}
		\underline{Example}
		\begin{eqnarray*}
			\begin{pmatrix}
				2 & 0 & 0\\
				8&-1&0\\
				7&6&8
			\end{pmatrix}
		\end{eqnarray*}
	\end{itemize}
\end{myDefinition}

\section{Identity Matrix}
\begin{myDefinition}
	\begin{eqnarray}
		a_{jj}=1&a_{jk}=0&\forall j\neq k
	\end{eqnarray}
	\begin{eqnarray}
		\mathbf{I}=
		\begin{pmatrix}
			1 & 0 & 0 & \cdots & 0\\
			0 & 1 & 0 & \cdots & 0\\
			0 & 0 & 1 & \cdots & 0\\
			\vdots &  &  & \cdots & \vdots\\
			0 & 0 & 0 &\cdots & 1
			
		\end{pmatrix}
	\end{eqnarray}
\end{myDefinition}
\begin{myDefinition}
	Diagonal Matrix
	\begin{eqnarray}
		\mathbf{A}=c\mathbf{I}
	\end{eqnarray}
\end{myDefinition}

\section{Linear Systems of Equations}
\subsection{Gauss Elimination}
\begin{myExample}
	\begin{eqnarray*}
		\begin{cases}
			x_1-x_2+x_3=0\\
			-x_1+x_2-x_3=0\\
			10x_2+25x_3=90\\
			20x_1+10x_2=80
		\end{cases}
		\\
		\Rightarrow
		\\
		\begin{pmatrix}
			1 & -1 & 1 & 0\\
			-1&1&-1&0\\
			0&10&25&90\\
			20&10&0&80
		\end{pmatrix}
	\end{eqnarray*}
	$l_1=l_2$, $\frac{l_3}{5}$ and $\frac{l4}{4}$.
	\begin{eqnarray*}
		\begin{pmatrix}
			1 & -1 & 1 & 0\\
			0&2&5&18\\
			2&1&0&8
		\end{pmatrix}
	\end{eqnarray*}
	$l_3-2l_1$
	\begin{eqnarray*}
		\begin{pmatrix}
			1 & -1 & 1 & 0\\
			0&2&5&18\\
			0&3&-2&8
		\end{pmatrix}
	\end{eqnarray*}
	$l_3-\frac{3}{2}l_2$
	\begin{eqnarray*}
		\begin{pmatrix}
			1 & -1 & 1 & 0\\
			0&2&5&18\\
			0&0&-\frac{19}{2}&-19
		\end{pmatrix}
	\end{eqnarray*}
	$-\frac{2}{19}l_3$
	\begin{eqnarray*}
		\begin{pmatrix}
			1 & -1 & 1 & 0\\
			0&2&5&18\\
			0&0&1&2
		\end{pmatrix}
	\end{eqnarray*}
	$l_2-5l_3$
	\begin{eqnarray*}
		\begin{pmatrix}
			1 & -1 & 1 & 0\\
			0&2&0&8\\
			0&0&1&2
		\end{pmatrix}
	\end{eqnarray*}
	$\frac{l_2}{2}$
	\begin{eqnarray*}
		\begin{pmatrix}
			1 & -1 & 1 & 0\\
			0&1&0&4\\
			0&0&1&2
		\end{pmatrix}
	\end{eqnarray*}
	$l_1+l_2-l_3$
	\begin{eqnarray*}
		\begin{pmatrix}
			1 & 0 & 0 & 2\\
			0&1&0&4\\
			0&0&1&2
		\end{pmatrix}
	\end{eqnarray*}
	Si on ré-écrit désormais le système, on obtient
	\begin{eqnarray*}
		\begin{cases}
			1\cdot x_1+0\cdot x_2+0\cdot x_3=2\\
			0\cdot x_1+1\cdot x_2+0\cdot x_3=4\\
			0\cdot x_2+0\cdot x_2+1\cdot x_3=2
		\end{cases}
	\end{eqnarray*}
	ce qui est équivalent à
	\begin{eqnarray*}
		\begin{cases}
			x_1=2\\
			x_2=4\\
			x_3=2
		\end{cases}
	\end{eqnarray*}
\begin{comment}
	\begin{eqnarray*}
		-\frac{19}{2}x_3=-19\Rightarrow x_3=2
		\\
		a
	\end{eqnarray*}
	

\end{comment}
\end{myExample}
\chapter{Linear Independance. Rank of a Matrix. Vector Space}
\begin{eqnarray}
	\label{eq:syslin}c_1\overrightarrow{a_1}+c_2\overrightarrow{a_2}+\dots+c_n\overrightarrow{a_n}=0
\end{eqnarray}
Si la seule solution du système \ref{eq:syslin} est $c_i=0$ pour $i=1,\dots,n$, alors les vecteurs $\overrightarrow a_i$ sont linéairement indépendants.

Le rang est le nombre de vecteurs indépendants d'une matrice.

Espace vectoriel $\mathbb R^n$ consiste de tous les vecteurs à $n$ composantes à la dimension $n$.

\section{Déterminants. Cramer's Rule}
\begin{eqnarray}
	\mathbf{D}=\det \mathbf{A}=\left|
	\begin{matrix}
		a_{11} & a_{12} & \cdots & a_{1n}
		\\
		\vdots & & & \vdots
		\\
		a_{n1} & a_{n2} & \cdots & a_{nn}
	\end{matrix}\right|
\end{eqnarray}
\begin{itemize}
	\item if $n=1$
	\begin{eqnarray}
		\det \mathbf{A}=a_{11}
	\end{eqnarray}
	\item if $n=2$
	\begin{eqnarray}
		\det \mathbf{A}=a_{11}a_{22}-a_{12}a_{21}
	\end{eqnarray}
	\item if $n=3$
	\begin{eqnarray}
		\det \mathbf{A}=a_{11}a_{22}a_{33}+a_{12}a_{23}a_{31}+a_{13}a_{21}a_{32}-a_{32}a_{23}a_{11}-a_{33}a_{21}a_{12}-a_{31}a_{22}a_{13}
	\end{eqnarray}
\end{itemize}

\begin{myDefinition}
	Matrice des signes
	\begin{eqnarray}
		\begin{pmatrix}
			+ & - & + & - & \cdots \\
			 - & + & - & + &  \cdots \\
			+ & - & + & - & \cdots \\
			 - & + & - & + & \cdots \\
			\vdots & & & \cdots
		\end{pmatrix}
	\end{eqnarray}
	This can also be expressed in the following manner
	\begin{eqnarray}
		a_{jk}=
		\begin{cases}
			+ & \text{if }(j+k)\text{ is even}
			\\
			- & \text{if }(j+k)\text{ is odd}
		\end{cases}
	\end{eqnarray}
\end{myDefinition}

\begin{myExample}
	Calculate
	\begin{eqnarray}
		\mathbf{D}=\det\mathbf{A}=
		\left|
		\begin{matrix}
			1 & 3 & 0\\
			2 & 6 & 4\\
			-1 & 0 &2
		\end{matrix}
		\right|
	\end{eqnarray}
	Using minors and cofactors one can write
	\begin{eqnarray*}
		\left|
		\begin{matrix}
			1 & 3 & 0\\
			2 & 6 & 4\\
			-1 & 0 &2
		\end{matrix}
		\right|
		=
		1
		\left|
		\begin{matrix}
			
			 6 & 4\\
			 0 &2
		\end{matrix}
		\right|
		-3
		\left|
		\begin{matrix}
		
			2  & 4\\
			-1  &2
		\end{matrix}
		\right|
		+0
		\left|
		\begin{matrix}
			
			2 & 6\\
			-1 & 0 
		\end{matrix}
		\right|
		=
		1\cdot12-3\cdot8+0=-12
	\end{eqnarray*}
\end{myExample}
\section{Evaluation of $\det$ by Reduction to Triangular Form}
\begin{myExample}
	\begin{eqnarray*}
		\mathbf{D}=\det \mathbf{A}=
		\left|
		\begin{matrix}
			2&0&-4&6\\
			4&5&1&0\\
			0&2&6&-1\\
			-3&8&9&1
		\end{matrix}
		\right|
	\end{eqnarray*}
	$l_2-2l_1$ and $l_4+\frac{3}{2}l_1$
	\begin{eqnarray*}
		\left|
		\begin{matrix}
			2&0&-4&6\\
			0&5&9&-12\\
			0&2&6&-1\\
			0&8&3&10
		\end{matrix}
		\right|
	\end{eqnarray*}
	$l_3-\frac{2}{5}l_2$ and $l_4+\frac{8}{5}l_2$
	\begin{eqnarray*}
		\left|
		\begin{matrix}
			2&0&-4&6\\
			0&5&9&-12\\
			0&0&2.4&3.8\\
			0&0&0&47.25
		\end{matrix}
		\right|=
		47.25
			\left|
			\begin{matrix}
				2&0&-4\\
				0&5&9\\
				0&0&2.4
			\end{matrix}
			\right|
			=
			47.25\cdot2.5
				\left|
				\begin{matrix}
					2&0\\
					0&5	
				\end{matrix}
				\right|
				=47.25\cdot2.5\cdot2\cdot5=1134
	\end{eqnarray*}
\end{myExample}

\begin{myDefinition}
	\begin{eqnarray}
		\det \mathbf{A}\neq0\Leftrightarrow \text{square matrix $n\times n$ has rank }n
	\end{eqnarray}
\end{myDefinition}
\section{Cramer's Rule}
If we have to solve the following system
\begin{eqnarray}
		\begin{cases}
			a_{11}x_1+a_{12}x_2+a_{13}x_3+\dots+a_{1m}x_m=b_1\\
			a_{21}x_1+a_{22}x_2+a_{23}x_3+\dots+a_{2m}x_m=b_2\\
			\vdots
			a_{n1}x_1+a_{n2}x_2+a_{n3}x_3+\dots+a_{nm}x_m=b_n
		\end{cases}
\end{eqnarray}
The solution is given by
\begin{eqnarray}
	\begin{cases}
		x_1=\frac{D_1}{D}
		\\
		x_2=\frac{D_2}{D}
		\\
		\vdots
		\\
		x_m=\frac{D_m}{D}
	\end{cases}
\end{eqnarray}

\begin{myExample}
	We want to solve 
	\begin{eqnarray*}
			\begin{cases}
				x_1-x_2+x_3=0\\
				2x_2+5x_3=18\\
				2x_1+x_2=8
			\end{cases}
	\end{eqnarray*}
	We write
	\begin{eqnarray*}
		\mathbf{\tilde{A}}=
		\begin{pmatrix}
			1 & -1 & 1 & 0\\
			0 & 2 & 5 & 18\\
			2 & 1 & 0 & 8
		\end{pmatrix}
		\\
		\mathbf{D}=
		\left|
		\begin{matrix}
			1 & -1 & 1\\
			0 & 2 & 5\\
			2 & 1 & 0
		\end{matrix}
		\right|=
		1
		\left|
		\begin{matrix}
			2 & 5\\
			1 & 0
		\end{matrix}
		\right|
		+2
		\left|
		\begin{matrix}
			-1 & 1\\
			2 & 5
		\end{matrix}
		\right|=(-5)+2\cdot(-7)=-19
		\\
		\mathbf{D_1}=
		\left|
		\begin{matrix}
			0 & -1 & 1\\
			18 & 2 & 5\\
			8 & 1 & 0
		\end{matrix}
		\right|
		=
		1
		\left|
		\begin{matrix}
			18 & 5\\
			8& 0
		\end{matrix}
		\right|
		+1
		\left|
		\begin{matrix}
			18 & 2\\
			8 & 1
		\end{matrix}
		\right|=(-40)+2=-38
		\\
		\mathbf{D_2}=
		\left|
		\begin{matrix}
			1 & 0 & 1\\
			0 & 18 & 5\\
			2 & 8 & 0
		\end{matrix}
		\right|
		=
		1
		\left|
		\begin{matrix}
			18 & 5\\
			8& 0
		\end{matrix}
		\right|
		+1
		\left|
		\begin{matrix}
			0&18 \\
			2&8 
		\end{matrix}
		\right|=(-40)+(-36)=-76
		\\
		\mathbf{D_3}=
		\left|
		\begin{matrix}
			1 & -1 & 0\\
			0 & 2 & 18\\
			2 & 1 & 8
		\end{matrix}
		\right|
		=
		1
		\left|
		\begin{matrix}
			2&18 \\
			1&8
		\end{matrix}
		\right|
		+1
		\left|
		\begin{matrix}
			0&18\\
			2&8
		\end{matrix}
		\right|=(-2)+(-36)=-38
	\end{eqnarray*}
	hence we obtain
	\begin{eqnarray*}
		\begin{cases}
			x_1=\mathbf{\frac{D_1}{D}}=2\\
			x_2=\mathbf{\frac{D_2}{D}}=4\\
			x_3=\mathbf{\frac{D_3}{D}}=2\\
		\end{cases}
	\end{eqnarray*}
\end{myExample}

\chapter{Inverse of a Matrix}
\begin{myDefinition}
	\begin{eqnarray}
		\mathbf{A}\cdot \mathbf{A^{-1}}=\mathbf{A^{-1}}\cdot \mathbf{A}=\mathbf{I}
	\end{eqnarray}
\end{myDefinition}

\section{Inverse of a Matrix: Gauss-Jordan Elimination}
\begin{myExample}
	\begin{eqnarray*}
		\mathbf{A}=
		\begin{pmatrix}
			-1 & 1 & 2\\
			3 & -1 & 1\\
			-1 & 3 & 4
		\end{pmatrix}
	\end{eqnarray*}
	
	\begin{eqnarray*}
		[\mathbf{A}\cdot \mathbf{I}]=
		\begin{pmatrix}
			-1 & 1 & 2 & 1 & 0 & 0\\
			3 & -1 & 1 & 0 & 1 & 0\\
			-1 & 3 & 4 & 0 & 0 & 1
		\end{pmatrix}
		\\
		\Rightarrow
		\begin{pmatrix}
			-1 & 1 & 2 & 1 & 0 & 0\\
			0 & 2 & 7 & 3 & 1 & 0\\
			0 & 2 & 2 & -1 & 0 & 1
		\end{pmatrix}
		\\
		\Rightarrow
		\begin{pmatrix}
			-1 & 1 & 2 & 1 & 0 & 0\\
			0 & 2 & 7 & 3 & 1 & 0\\
			0 & 0 & -5 & -4 & 1 & 1
		\end{pmatrix}
		\\\\
		\Rightarrow \Rightarrow \Rightarrow
		\\\\
		\Rightarrow
		\begin{pmatrix}
			1 & 0 & 0 & -0.7 & 0.2 & 0.3\\
			0 & 1 & 0 & -1.3 & -0.2 & 0.7\\
			0 & 0 & 1 & 0.8 & 0.2 & -0.2
		\end{pmatrix}
	\end{eqnarray*}
	
	\begin{eqnarray*}
		\mathbf{A^{-1}}=\begin{pmatrix}
			 -0.7 & 0.2 & 0.3\\
			 -1.3 & -0.2 & 0.7\\
			 0.8 & 0.2 & -0.2
		\end{pmatrix}
	\end{eqnarray*}
\end{myExample}
\begin{myTheorem}
	\begin{eqnarray}
		\mathbf{A^{-1}}=\frac{1}{\det \mathbf{A}}[c_{jk}]^{T}
		=\frac{1}{\det \mathbf{A}}
		\begin{bmatrix}
			c_{11} & c_{12} & \cdots & c_{n1}\\
			\vdots & & & \vdots\\
			c_{n1} & c_{n2} & \cdots & c_{nm}
		\end{bmatrix}
	\end{eqnarray}
\end{myTheorem}

\begin{myExample}
	\begin{eqnarray*}
		\textbf{A}=
		\begin{bmatrix}
			a_{11} &a_{12}\\
			a_{21} & a_{22}
		\end{bmatrix}
		\\
		\mathbf{C}=
		\begin{bmatrix}
			a_{22} &-a_{21}\\
			-a_{12} & a_{11}
		\end{bmatrix}
		\\
		\mathbf{C^T}=
		\begin{bmatrix}
			a_{22} &-a_{12}\\
			-a_{21} & a_{11}
		\end{bmatrix}
		\\
		\mathbf{A^{-1}}=
		\frac{1}{\det \mathbf{A}}
		\begin{bmatrix}
			a_{22} &-a_{12}\\
			-a_{21} & a_{11}
		\end{bmatrix}
	\end{eqnarray*}	
\end{myExample}

\begin{myDefinition}
	\begin{eqnarray}
		\mathbf{A}\overrightarrow x=\overrightarrow b
		\Leftrightarrow
		\overrightarrow x=\mathbf{A^{-1}}\overrightarrow b
	\end{eqnarray}
\end{myDefinition}

\chapter{Matrix Eigenvalue Problem}
\begin{myDefinition}
	\begin{eqnarray}
		\textbf{A}\overrightarrow x=\lambda\overrightarrow x
	\end{eqnarray}
	$\Rightarrow \textbf{A}\overrightarrow x\propto\overrightarrow x$
\end{myDefinition}

\begin{myExample}
	\begin{eqnarray*}
		\mathbf{A}=
		\begin{pmatrix}
			-5 & 2\\
			2 & -2
		\end{pmatrix}
	\end{eqnarray*}
	
	\begin{eqnarray*}
			\mathbf{A}\overrightarrow x=
			\begin{pmatrix}
				-5 & 2\\
				2 & -2
			\end{pmatrix}
			\begin{pmatrix}
				x_1\\x_2
			\end{pmatrix}
			=\lambda\overrightarrow x
			\\\Rightarrow
			\begin{cases}
				-5x_1+2x_2=\lambda x_1\\
				2x_1-2x_2=\lambda x_2
			\end{cases}
	\end{eqnarray*}
	now we write
	\begin{eqnarray*}
		|\mathbf{A}-\lambda \mathbf{I}|\overrightarrow x=0
		\\
		=
		\begin{pmatrix}
			-5-\lambda&2\\
			2&-2-\lambda
		\end{pmatrix}
		\begin{pmatrix}
			x_1\\x_2
		\end{pmatrix}
	\end{eqnarray*}
	
	\begin{eqnarray*}
		\det |\mathbf{A}-\lambda \mathbf{I}|\\
		=(5+\lambda)(2+\lambda)-4=(\lambda+1)(\lambda+6)
	\end{eqnarray*}
	Thus we find our \textbf{eigenvalues}:
	\begin{eqnarray*}
		\begin{cases}
			\lambda_1=-1\\
			\lambda_2=-6
		\end{cases}
	\end{eqnarray*}

\textbf{Eigenvectors}
with $\lambda_1$
\begin{eqnarray*}
	\begin{pmatrix}
		-5+1&2\\
		2&-2+1
	\end{pmatrix}
	\begin{pmatrix}
		x_1\\x_2
	\end{pmatrix}
	=
	\begin{pmatrix}
		-4&2\\
		2&-1
	\end{pmatrix}
	\begin{pmatrix}
		x_1\\x_2
	\end{pmatrix}
	=
	0
	\\
	\Rightarrow
	2x_1=x_2
	\\
	\begin{bmatrix}
		x_1\\x_2
	\end{bmatrix}
	=c\begin{bmatrix}
		1\\2
	\end{bmatrix}
	\\\overrightarrow v_2=\begin{bmatrix}
		1\\2
	\end{bmatrix}
\end{eqnarray*}
with $\lambda_1$
\begin{eqnarray*}
	\begin{pmatrix}
		-5+6&2\\
		2&-2+6
	\end{pmatrix}
	\begin{pmatrix}
		x_1\\x_2
	\end{pmatrix}
	=
	\begin{pmatrix}
		1&2\\
		2&4
	\end{pmatrix}
	\begin{pmatrix}
		x_1\\x_2
	\end{pmatrix}
	=
	0
	\\
	\Rightarrow
	x_1=-2x_2
	\\
	\begin{bmatrix}
		x_1\\x_2
	\end{bmatrix}
	=c\begin{bmatrix}
		-2\\1
	\end{bmatrix}
	\\\overrightarrow v_2=\begin{bmatrix}
		-2\\1
	\end{bmatrix}
\end{eqnarray*}


\end{myExample}
\section{Matrice rotation}
Counterclockwise rotation of $\theta$.
\begin{eqnarray}
	\begin{bmatrix}	
		\cos{(\theta)} & -\sin{(\theta)}\\
		\sin{(\theta)} & \cos{(\theta)}
	\end{bmatrix}
	\begin{bmatrix}
		x\\
		y
	\end{bmatrix}
	=
	\begin{bmatrix}
		\hat x\\
		\hat y
	\end{bmatrix}
\end{eqnarray}