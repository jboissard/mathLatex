%!TEX root = /Users/Johan/Documents/University/Math/Gymnase/book/book.tex
\chapter{Dérivée}
\section{Fonction composée}
\begin{myDefinition}
	$y=f(u)$ $u=g(x)$ $y=f(g(x))=(f\circ g)(x)$
\end{myDefinition}
\begin{myExample}
	
	
	$y=f(u)=\sqrt{u}$ 
	\\
	$u=g(x)=x^2+1$
	\\
	$y=(f\circ g)(x)$
	Calculer $f'(x)$.
	
	\begin{eqnarray*}
		f'(u)=\frac{1}{2}u^{-\frac{1}{2}}
		\\
		g'(x)=2x
		\\
		(f\circ g)'(x)=f'(g(x))g'(x)=\frac{x}{\sqrt{x^2+1}}
	\end{eqnarray*}
\end{myExample}
\begin{eqnarray}
	f'(x)=\frac{dy}{dx}(x)=\lim_{\Delta x\rightarrow 0}\frac{f(x+\Delta x)-f(x)}{\Delta x}
\end{eqnarray}
\begin{eqnarray}
	\frac{d}{dx}(af(x))=a\frac{d}{dx}f(x)
	\\
	\frac{d}{dx}\left(f(x)+g(x)\right)=	\frac{d}{dx}f(x)+\frac{d}{dx}g(x)
	\\
	\frac{d}{dx}\left((f\circ g)(x)\right)=(f \circ g)' = (f'\circ g) \cdot g'
	\\
	(f\cdot g)'=f'\cdot g+f\cdot g'
	\\
	\frac{f}{g}=\frac{f'\cdot g-f\cdot g'}{g^2}
\end{eqnarray}

\textbf{Exemple:}

\begin{eqnarray*}
	\left(7\frac{x^2+x+5}{e^{2x}}\right)'
\end{eqnarray*}
On peut récrire sous la forme suivante
\begin{eqnarray}
	\left(a\frac{f}{g}\right)'=a\left(\frac{f}{g}\right)'=a\frac{f'\cdot g-f\cdot g'}{g^2}
\end{eqnarray}
où $a=7$, $f=x^2+x+5$ et $g=e^{2x}$. 

On a $f'(x)=2x+1$ et $g'(x)=2e^{2x}$ et obtient donc:
\begin{eqnarray*}
	\left(7\frac{x^2+x+5}{e^{2x}}\right)'=7\frac{e^{2x}((2x+1)-2\cdot(x^2+x+5))}{4e^{4x}}
	=-\frac{7}{4}\frac{2x^2+9}{e^{2x}}
\end{eqnarray*}

Pour $y=f(x)\Rightarrow dy=f'(x)\Delta x=f'(x)dx$, on a $dx=\Delta x$ \underline{\textbf{mais}} $dy\neq\Delta y$!

\begin{eqnarray}
	\Delta y=f(x+\Delta x)-f(x)\text{ ou }	\Delta y+f(x)=f(x+\Delta x)
\end{eqnarray}

\begin{myExample}
	$y=3x^2-5$, $x=2$ et $\Delta x=0.1$
	Calculer $\Delta y$ et $dy$.
	
	Tout d'abord on remarque que $\Delta x=dx$ et $f'(x)=6x$.
	\begin{eqnarray*}
		\Delta y=f(x+\Delta x)-f(x)=f(2.1)-f(2)=1.23
	\end{eqnarray*}
	et
	\begin{eqnarray*}
		dy=\frac{dy}{dx}dx=f'(x)dx=f'(x)\Delta x=f'(2)\cdot0.1=1.2
	\end{eqnarray*}
	
\end{myExample}

\section{Dérivée de fonctions implicites}
si $y^4+3y-4x^3=5x+1$ défini sur une fonction implicite $f(x)$, alors on dérive tous les termes par $x$ et on factorise par $y'$.
\begin{eqnarray*}
	(y^4)'+(3y)'-(4x^3)'=(5x)'+(1)'=
	\\
	4y^3y'+3y'-12x^2=5
	\\
	y'(4y^3+3)=12x^2+5
\end{eqnarray*}
et
\begin{eqnarray*}
	y'=f'(x)=\frac{12x^2+5}{4(f(x))^3+3}
\end{eqnarray*}
\begin{myExample}
	Taux liés
	
	Soit $x^3-2y^2+5x=16$, $\frac{dx}{dt}=4$, $x=2$, $y=-1$ et $\frac{dy}{dt}=?$
	\begin{eqnarray*}
		\frac{d}{dt}(x^3)'-\frac{d}{dt}(2y^2)'+\frac{d}{dt}(5x)'=\frac{d}{dt}(16)
		\\
		3x^2\frac{dx}{dt}-4y\frac{dy}{dt}+5\frac{dx}{dt}=0
		\\
		3\cdot2^2\cdot4-4\cdot(-1)\frac{dy}{dt}+5\cdot 4=0
		\\
		\Rightarrow \frac{dy}{dt}=\frac{-20-48}{4}=-17
	\end{eqnarray*}
\end{myExample}
\chapter{Applications de la dérivée}
\section{Extremums}
Si $f(x)$ est continue et que $f'(x)=0$ a une solution cette dernière est un point critique (max, min, point de selle).

\begin{myExample}
	\begin{eqnarray*}
		f(x)=2\sin{(x)}+\cos{(2x)}
		\\
		f'(x)=2\cos{(x)}-2\sin{(2x)}=2\cos{(x)}-4\sin{(x)}\cos{(x)}=2\cos{(x)}(1-2\sin{(x)})
	\end{eqnarray*}
	Il nous faut donc trouver les solutions du système d'équation suivant
	\begin{eqnarray*}
		\begin{cases}
			\cos{(x)}=0
			\\
			1-2\sin{(x)}=0 \Leftrightarrow \frac{1}{2}=\sin{(x)}
		\end{cases}	
	\end{eqnarray*}
	La solution de la première équation est $x=(1+2k)\pi$, la résolution de la deuxième équation est un peu plus délicate mais on trouve
	\begin{eqnarray*}
		x=\begin{cases}
			\frac{\pi}{6}+2k\pi
			\\
			-\frac{\pi}{6}+(1+2k)\pi
		\end{cases}
	\end{eqnarray*}
	Les extremums de $f(x)$ sont donc donnés par
	\begin{eqnarray*}
		x=\begin{cases}
			(1+2k)\pi
			\\
			\frac{\pi}{6}+2k\pi
			\\
			-\frac{\pi}{6}+(1+2k)\pi
		\end{cases}
	\end{eqnarray*}
	avec $k\in\mathbb Z$.
\end{myExample} 

\section{Théorème des accroissements finis}
Si $f$ est une fonction continue sur l'intervalle fermé $[a,b]$ est dérivable sur l'intervalle ouvert $]a,b[$, tel que $a<b$ et $f(a)=f(b)$. $f'(c)=0$, avec $c\in]a,b[$, en au moins un point de $]a,b[$.

\begin{myTheorem}\textbf{Théorème de accroissements finis}
	Si $f$ est dérivable sur $[a,b]$ alors \begin{eqnarray}
		f'(c)=\frac{f(b)-f(a)}{b-a}
	\end{eqnarray} 
\end{myTheorem}

\section{Test de la dérivée première}
Soit $f$ continue sur $[a,b]$ et dérivable sur $]a,b[$
\begin{enumerate}
	\item si $f'(x)>0$  fonction croissante
	\item si $f'(x)<0$  fonction décroissante
	\item si $f'(x)=0$  point critique
\end{enumerate}
\section{Concavité et test de la dérivée seconde}
	\begin{enumerate}
		\item si f''(x)>0 convexe
		\item si f''(x)<0 non convexe (concave)
		\item si f''(x)=0 point d'inflexion (Wendepunkt)
	\end{enumerate}
\section{Problème d'optimisation}
\begin{enumerate}
	\item créer une fonction représentant la donnée du problème
	\item dériver
	\item déterminer quelle genre d'optimum on trouve et quel genre d'optimum on cherche. Attention aux "rand"max qui sont déterminés par la donnée du problème et qui ne correspondent pas forcément à l'optimum trouvé.
\end{enumerate}

\begin{myExample}
	$f(x)=(32-2x)x$, $f'(x)=32-4x$, $\hat x=8$
\end{myExample}

\section{Méthode de Newton}	
La méthode de Newton est une méthode numérique permettant de trouver les zéros d'une fonction, pour plus de détails se référer au cours d'analyse numérique.

Soit $f(x)$ une fonction continue sur $[a,b]$, une approximation de $f(x)=0$ peut être donnée par
\begin{eqnarray}
	x^{(k+1)}=x^{(k)}-\frac{f(x^{(k)})}{f'(x^{(k)})}
\end{eqnarray}
où $x^{(0)}\in[a,b]$ est choisi arbitrairement.

\begin{myExample}
On cherche les zéros de	$f(x)=x-\cos{(x)}$.

On a $f'(x)=1-\sin{(x)}$ la méthode s'écrit donc
\begin{eqnarray*}
	x^{(k+1)}=x^{(k)}-\frac{x^{(k)}-\cos{(x^{(k)})}}{1-\sin{(x^{(k)})}}
\end{eqnarray*}
en prenant $x^{(0)}=0.8$ on a
\begin{eqnarray*}
	x^{(0)}=& &0.8
	\\
	x^{(1)}=&0.8-\frac{0.8-\cos{(0.8)}}{1-\sin{(0.8)}}=&0.74
	\\
	x^{(2)}=&\cdots&0.739
	\\
	x^{(3)}=&\cdots&0.739
	\\
	x^{(4)}=&\cdots&0.739085133\dots
\end{eqnarray*}
\end{myExample}

\section{The differential operator: ${\rm d}$}
The $\rm d$ operator indicates an infinitely small interval.

\begin{equation}
	\lim_{\Delta x\rightarrow0}\Delta x={\rm d}x
\end{equation}
\section{Operator Algebra}
The operator $\rm d$ followed by its associated variable can be treated as a common variable and allows to find interesting and practical identities. Be aware that the operator is not dissociable from its attached variable; not allowed to divide by ${\rm d}$! For instance
\begin{eqnarray}
	\frac{{\rm d} x^2}{{\rm d}x}=2x\\
	\Rightarrow
	{\rm d}x^2 =2x{\rm d}x
\end{eqnarray}

or

\begin{eqnarray}
	\frac{{\rm d} \ln{(x)}}{{\rm d}x}=\frac{1}{x}\\
	\Rightarrow
	{\rm d} \ln{(x)} =\frac{1}{x}{\rm d}x
\end{eqnarray}

The latter proves that a percentage change can simply be expressed as the derivative of the $\log$ function.
\begin{equation}
	\frac{\dot x}{x} = \frac{{\rm d} \ln{(x)}}{{\rm d}t}
\end{equation}
In discrete time, we find the following approximation for small changes
\begin{equation}
	\frac{x_1-x_0}{x_0}\approx\ln{\left(\frac{x_1}{x_0}\right)}
\end{equation}

More generally, one can state
\begin{equation}
	{\rm d}f(x) = \left(\frac{{\rm d}f(x)}{{\rm d}x}\right){\rm d}x=f'(x){\rm d}x
\end{equation}


\begin{myExample}
	In physics it is know that work is the line integral of the force:
	\begin{equation}
		W_\Gamma = \int_{\Gamma} Fd\gamma.
	\end{equation}
	
	So the instantaneous force can be retrieved by differentiating $W$ at $\mathbf{x}$.
	
	When the $W$ is only potential energy, i.e. $W=U=mgh = mg(x-0)$ it is very easy to retrieve the gravity force.
	
	\begin{equation}
		F = \frac{{\rm d} U}{{\rm d} x} = mg.
	\end{equation}
	
	
	Now, if we do the same exercie with work that comprises only kinetic energy, i.e. $W=K=\frac{1}{2}mv^2$, one can retrieve Newton's second law by using operator algebra.
	
	\begin{enumerate}
		\item Rewrite $v$ in the differental form
		\begin{equation}
			K = \frac{1}{2}m\left(\frac{{\rm d}x}{{\rm d} t}\right)^2
		\end{equation}
		
		\item apply differentiation rigorously and methodically, also to operators
		\begin{eqnarray}
			\frac{\rm d K}{{\rm d}x} 
			&=& \frac{\rm d}{{\rm d}x}\left(\frac{1}{2}m\left(\frac{{\rm d}x}{{\rm d}t}\right)^2\right)\\
			&=& \frac{1}{2}m\frac{\rm d}{{\rm d}x}\left(\left(\frac{{\rm d}x}{{\rm d}t}\right)^2\right)\\
			&=& \frac{m}{2{\rm d}t^2}\frac{\rm d}{{\rm d}x}\left(\left({\rm d}x\right)^2\right)\\
			&=& \frac{m}{2{\rm d}t^2}\frac{2{\rm d}x}{{\rm d}x}{\rm d}{\rm d}x\\
			&=& m \frac{{\rm d}^2x}{{\rm d}t^2}\\
			&=& ma
		\end{eqnarray}
	\end{enumerate}
\end{myExample}


