\chapter{Suites}
Une \textbf{suite} est une application de $\mathbb N$ (ou d'une partie de $\mathbb N$) vers $\mathbb R$. L'image de $n\in\mathbb N$ par cette application, notée $u_n$, converge vers un nombre réel $a$ si $\lim_{n\rightarrow+\infty}u_n=a$.

\begin{enumerate}
	\item Une suite croissante et majorée converge
	\item Une suite décroissante et minorée converge
\end{enumerate}

\section{Suite arithmétique}
La suite $u_1,u_2,u_3,\dots$ est une suite \textbf{arithmétique} de raison $r$ si, pour tout $n\in\mathbb N^*, u_{n+1}=u_n+r$.

\begin{eqnarray}
	u_n=u_1+(n-1)r\\
	\sum_{i=1}^{n}u_i=n\frac{u_1+u_n}{2}
\end{eqnarray}

\section{Suite géométrique}
La suite $u_1,u_2,u_3,\dots$ est une suite \textbf{géométrique} de raison $r$ si, pour tout $n\in\mathbb N^*, u_{n+1}=r\cdot u_n$.
\begin{eqnarray}
	u_n=u_1\cdot r^{n-1}\\
	a \sum_{i=0}^{n} q^{i} =a \frac{1-q^{n+1}}{1-q}\\
	\lim_{n\rightarrow+\infty}\sum_{i=0}^{n} q^{i}=\frac{1}{1-q}&|q|<1
\end{eqnarray}

\subsection{Proof}
One very elegant proof is to consider the following equation
\begin{equation*}
	\label{eq:beginprooffseriegeom}
	x = 1+ax
\end{equation*}
we know $x= \frac{1}{1-a}$

We substitute $1+ax$ to $x$ on the RHS of the equation which gives
\begin{eqnarray*}
	x &=& 1+ax\\
	&=& 1 + a(1+ax)\\
	&=& 1 + a(1+a(1+x))\\
	&=& 1 + a + a^2 + a^3 + ... +a^n + a^{n+1}x\\
	&=& \sum_{k=0}^n a^k + a^{n+1}x
\end{eqnarray*}
which leads to
\begin{eqnarray*}
	\sum_{k=0}^n a^k &=&   x-a^{n+1}x= x\left(1-a^{n+1}\right)
\end{eqnarray*}
since we know from eq \ref{eq:beginprooffseriegeom} $x=\frac{1}{1-a}$, we subsitute and find
\begin{equation*}
	\sum_{k=0}^n a^k =  \left(1-a^{n+1}\right)\frac{1}{1-a}
\end{equation*}
which is the geometric serie.

\chapter{Series}
La série de termes $u_k$ convege si la suite $s_n=\sum_{k=1}^nu_k$ converge. La limite de cette suite, notée $\sum_{k=1}^\infty u_k$, est la somme de la série.
\section{Series convergentes et divergentes}
\subsection{Power serie}
Power series are defined as follows:
\begin{eqnarray}
	\sum_{n=0}^{\infty}a_nx^n
\end{eqnarray}

Two very important properties applies to those series
\begin{eqnarray}
	\sum_{n=0}^{\infty}\frac{d}{dx}a_nx^n=\frac{d}{dx}\sum_{n=0}^{\infty}a_nx^n
\end{eqnarray}
and
\begin{eqnarray}
	\sum_{n=0}^{\infty}\int_0^{x}a_nx^ndx=\int_0^x\left(\sum_{n=0}^{\infty}a_nx^n\right)dx
\end{eqnarray}

Here is one example
\begin{eqnarray*}
	\sum_{n=0}^{\infty}\frac{z^{-n}}{n}=\sum_{n=0}^{\infty}\int_0^z-z^{-n-1}dz=\int_0^z\underbrace{\left(\sum_{n=0}^{\infty}-z^{-n-1}\right)}_{-\frac{1}{1-z}}dz=-\int_0^z\frac{1}{1-z}dz=1-\ln{(1-z)}
\end{eqnarray*}


\subsection{Série géométrique}
\subsubsection{Somme de nombre consécutifs}
\begin{eqnarray}
	\sum_{i=0}^{n}i=\sum_{i=1}^ni=\frac{1}{2}n(n+1)
\end{eqnarray}
\subsubsection{Somme de carrés}
\begin{eqnarray}
	\sum_{i=0}^{n}i^2=\sum_{i=1}^ni^2=\frac{1}{3}n^3+\frac{1}{2}n^2+\frac{1}{6}n=\frac{1}{3}n\left(n+1\right)\left(n+\frac{1}{2}\right)
\end{eqnarray}
\begin{eqnarray}
	\sum_{i=0}^{n}i^3=\sum_{i=1}^ni^3=\frac{1}{4}n^2\left(n+1\right)^2
\end{eqnarray}
\begin{eqnarray}
	\sum_{i=0}^{n}i^4=\sum_{i=1}^ni^4=\frac{1}{4}n^2\left(n+1\right)^2
\end{eqnarray}
\subsection{Série téléscopique}
\section{Séries à termes positifs}
\subsection{Série de Riemann}
\section{Les tests d'Alembert et les racines}
\subsection{Critèredu quotient (Alembert)}
Si $\lim_{k\rightarrow+\infty}\frac{u_{k+1}}{u_k}=c$ et $c \begin{cases}
	c<1 \text{ , la série converge}\\c>1 \text{ , la série diverge}
\end{cases}$

\subsection{Critère de la racine (Cauchy)}
Si $\lim_{k\rightarrow+\infty}\sqrt{u_k}=\lim_{k\rightarrow+\infty}u_k^{1/k}=c$ et $\begin{cases}
	c<1\text{, la série converge}\\
	c>1\text{, la série diverge}
\end{cases}$
\section{Séries alternées et convergence absolue}
\section{Fonctions représentées par une série entière}
\section{La série du binôme}
\section{Série entière}
Une série de terme général $u_k$ est appelée \textbf{série entière} si $u_k=a_kx^k$ avec $a_k\in\mathbb R$.

Rayon de convergence: 
\begin{eqnarray}
	r=\lim_{k\rightarrow+\infty}\left|\frac{a_k}{a_k+1}\right|\\
	r=\frac{1}{\lim_{k\rightarrow+\infty}|a_k|^{1/k}}
\end{eqnarray}
La série entière de terme $a_x^k \begin{cases}
	\text{converge si }|x|<r\\
	\text{diverge si }|x|>r
\end{cases}$
Si $|x|=r$, il y a un doute.

Si $r=+\infty$, alors la série entière converge pour tout réel $x$.
\section{Exemples de séries}
\subsection{Exemples de séries divergentes}
\begin{eqnarray}
	1+\frac{1}{2}+\frac{1}{3}+\frac{1}{4}+\frac{1}{5}+\dots=\sum_{i=1}^\infty\frac{1}{i}=+\infty&\text{Série harmonique}\\
	1+\frac{1}{\sqrt 2}+\frac{1}{\sqrt 3}+\frac{1}{\sqrt 4}+\frac{1}{\sqrt 5}+\dots=\sum_{i=1}^\infty\frac{1}{\sqrt i}=+\infty
\end{eqnarray}
et plus généralement
\begin{eqnarray}
	1+\frac{1}{2^\alpha}+\frac{1}{3^\alpha}+\frac{1}{4^\alpha}+\frac{1}{5^\alpha}+\dots=\sum_{i=1}^\infty\frac{1}{i^\alpha}=+\infty&\text{si $\alpha\leq1$}
\end{eqnarray}
\begin{eqnarray}
	1+r+r^2+r^3+\dots+r^k+\dots=+\infty&\text{si $r\geq1$}
\end{eqnarray}
\subsection{Exemples de séries convergentes}
\begin{eqnarray}
	1+\frac{1}{1!}+\frac{1}{2!}+\frac{1}{3!}+\dots+\frac{1}{k!}+\dots=\sum_{k=0}^\infty\frac{1}{k!}=e\\
	1+\frac{1}{2}+\frac{1}{4}+\frac{1}{8}+\dots=\sum_{k=0}^\infty\frac{1}{2^k}=2\\
	1+r+r^2+r^3+\dots=\sum_{i=1}^{n}r^i=\frac{1}{1-r}&|r|<1
\end{eqnarray}
\begin{eqnarray}
	1+\frac{1}{2^\alpha}+\frac{1}{3^\alpha}+\dots=\sum_{k=0}^\infty\frac{1}{k^\alpha}=\zeta(\alpha)&\text{ fonction zêta de Riemann}\\
	1+\frac{1}{4}+\frac{1}{9}+\frac{1}{25}+\dots=\zeta(2)=\frac{\pi^2}{6}\\
	1+\frac{1}{16}+\frac{1}{81}+\frac{1}{256}+\dots=\zeta(4)=\frac{\pi^4}{90}\\
	1+\frac{1}{9}+\frac{1}{25}\frac{1}{49}+\dots=\sum_{k=0}^\infty\frac{1}{(2k+1)^2}=\frac{\pi^2}{8}\\
	1+\frac{1}{81}+\frac{1}{625}\frac{1}{2401}+\dots=\sum_{k=0}^\infty\frac{1}{(2k+1)^4}=\frac{\pi^4}{96}
\end{eqnarray}

\begin{eqnarray}
	\frac{1}{1\cdot2}+\frac{1}{2\cdot3}+\frac{1}{3\cdot4}+\dots=\sum_{k=1}^\infty\frac{1}{k\cdot(k+1)}=1\\
	\frac{1}{1\cdot3}+\frac{1}{3\cdot5}+\frac{1}{5\cdot7}+\dots=\sum_{k=1}^\infty\frac{1}{(2k-1)\cdot(2k+1)}=\frac{1}{2}\\
	\frac{1}{1\cdot3}+\frac{1}{2\cdot4}+\frac{1}{3\cdot5}+\dots=\sum_{k=2}^\infty\frac{1}{(k-1)\cdot(k+1)}=\frac{3}{4}\\
	\frac{1}{3\cdot5}+\frac{1}{7\cdot9}+\frac{1}{11\cdot13}+\dots=\sum_{k=1}^\infty\frac{1}{(4k-1)\cdot(4k+1)}=\frac{1}{2}-\frac{\pi}{8}
\end{eqnarray}
\begin{eqnarray}
	1-\frac{1}{1!}+\frac{1}{2!}-\frac{1}{3!}+\dots=\sum_{k=0}^\infty(-1)^k\frac{1}{k!}=\frac{1}{e}\\
	1-\frac{1}{2}+\frac{1}{3}-\frac{1}{4}+\dots=\sum_{k=0}^\infty(-1)^k\frac{1}{k+1}=\ln{2}&\text{Série harmonique alternée}\\
	1-\frac{1}{3}+\frac{1}{5}-\frac{1}{7}+\dots=\sum_{k=0}^\infty(-1)^k\frac{1}{2k+1}=\frac{\pi}{4}\\
	1-\frac{1}{4}+\frac{1}{9}-\frac{1}{16}+\dots=\sum_{k=0}^\infty(-1)^k\frac{1}{(k+1)^2}=\frac{\pi}{4}
\end{eqnarray}

\section{Formule de Taylor d'ordre $n$}
On note $f$ une fonction $n+1$ fois continuement dérivable dans un intervalle ouvert $I$ contenant $a$.

Pour tout $x\in I$:

\begin{eqnarray}
	f(x)=f(a)+\frac{1}{1!}f'(a)(x-a)+\frac{1}{2!}f''(a)(x-a)^2+\dots+\frac{1}{n!}f^{(n)}(a)(x-a)^n+R_n(x)
\end{eqnarray}
avec \begin{eqnarray}
	R_n(x)=\int_a^x\frac{1}{n!}f^{(n+1)}(t)(x-t)^ndt=\frac{1}{(n+1)!}f^{(n+1)}(c)(x-a)^{n+1}
\end{eqnarray} où $c$ est compris entre $a$ et $x$.

Estimation du reste: $|R_n(x)|\leq\frac{|x-a|^{n+1}}{(n+1)!}\mathrm{sup}_{t\in I}{|f^{(n+1)}(t)|}$.

Si $\lim_{n\rightarrow+\infty}R_n(x)=0$, alors $f(x)=\sum_{k=0}^\infty\frac{1}{k!}f^{(k)}(a)(x-a)^k$ et la série de terme $\frac{1}{k!}f^{(k)}(a)(x-a)^k$ est appelée \textbf{série de Taylor de $f$ contrée en $a$}.
\subsection{Formule de MacLaurin}
Si $a=0$, on obient la formule de MacLaurin:
\begin{eqnarray}
	f(x)=f(0)+\frac{1}{1!}f'(0)x+\frac{1}{2!}f''(0)x^2+\dots+\frac{1}{n!}f^{(n)}(0)x^n+R_n(x)
\end{eqnarray}