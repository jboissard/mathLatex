\chapter{Elementary Algebra}

\section{Identities}
\begin{eqnarray}
	(a+b)^2=a^2+2ab+b^2\\
	(a-b)^2=a^2-2ab+b^2\\
	(a+b)^3=a^3+3a^2b+3ab^2+b^3\\
	(a-b)^3=a^3-3a^2b+3ab^2+b^3\\
	(a+b)^n={n\choose 0}a^n+{n\choose 1}a^{n-1}b+{n\choose 2}a^{n-2}b^2+\dots+{n\choose k}a^{n-k}b^k+\dots+{n\choose n}b^n\\
	a^2-b^2=(a-b)(a+b)\\
	a^2+b^2=(a-ib)(a+ib)\\
	a^3-b^3=(a-b)(a^2+ab+b^2)=(a-b)\left(a+(1+i\sqrt3)\frac{b}{2}\right)\left(a+(1-i\sqrt3)\frac{b}{2}\right)\\
	a^3-b^3=(a+b)(a^2-ab+b^2)=(a-b)\left(a-(1+i\sqrt3)\frac{b}{2}\right)\left(a-(1-i\sqrt3)\frac{b}{2}\right)\\
	a^n-b^n=(a-b)(a^{n-1}+a^{n-2}b+\dots+ab^{n-2}+b^{n-1}\\
	(a+b+c)^2=a^2+b^2+c^2+2ab+2bc+2ac
\end{eqnarray}

\section{Powers and roots}
$a, b>0$ and $\sqrt[n]{a}$ is only defined for $n\in\mathbb N^*$.
\begin{eqnarray}
	a^0=1\\
	a^p=a\cdot a^{p-1}\\
	a^{-q}=\frac{1}{a^q}\\
	a^{\frac{1}{q}}=\sqrt[q]{a}\\
	a^{\frac{p}{q}}=\sqrt[q]{a^p}\\
	a^pa^q=a^{p+q}\\
	\frac{a^p}{a^q}=a^{p-q}\\
	(a^p)^q=a^{p\cdot q}\\
	a^pb^p=(ab)^p\\
	\frac{a^p}{b^p}=\left(\frac{a}{b}\right)^p
	\left(\sqrt[n]{a}\right)^n=a\\
	\left(\sqrt[q]{a}\right)^p=\sqrt[q]{a^p}\\
	\sqrt[p]{\sqrt[q]{a}}=\sqrt[pq]{a}=a^{\frac{1}{pq}}\\
	\sqrt[p]{a}\sqrt[p]{b}=\sqrt[p]{ab}=(ab)^{\frac{1}{p}}\\
	\frac{\sqrt[p]{a}}{\sqrt[p]{b}}=\sqrt[p]{\frac{a}{b}}=\frac{a}{b}^{\frac{1}{p}}
\end{eqnarray}

\section{Absolute Value}
\begin{eqnarray}
	|a|=\begin{cases}
		a & a\geq0\\
		-a & a<0
	\end{cases}
\end{eqnarray}

\begin{eqnarray}
	|a|\cdot |b|=|ab|\\
	\frac{|a|}{|b|}=|\frac{a}{b}|\\
	\sqrt{a^2}=|a|\\
	\left ||a|-|b|\right |\leq |a+b|\leq |a| + |b|
\end{eqnarray}
\section{Means}
\begin{tabular}{|l|c|c|}
	\hline
	\textbf{Mean} & \textbf{of two numbers $a_1$ and $a_2$} & \textbf{of $n$ numbers $a_1, a_2,\dots$}\\
	\hline
	Arithmetic (A) & $\frac{a_1+a_2}{2}$ & $\frac{\sum_{i=1}^na_i}{n}$\\
	Weighted (W)& $\frac{\lambda_1a_1+\lambda_2a_2}{\lambda_1+\lambda_2}$ & $\frac{\sum_{i=1}^n\lambda_ia_i}{\sum_{i=1}^n\lambda_i}$\\
	Geometric (G)& $\sqrt{a_1a_2}$ & $\sqrt[n]{\prod_{i=1}^na_i}$\\
	Harmonic (H)& $\frac{2}{\frac{1}{a_1}+\frac{1}{a_23}}$ & $\frac{n}{{\sum_{i=1}^n\frac{1}{a_i}}}$\\
	Quadratic (Q)& $\sqrt{\frac{a_1^2+a_2^2}{2}}$ & $\sqrt{\frac{\sum_{i=1}^na_i^2}{n}}$\\
	\hline
\end{tabular}

Property
\begin{eqnarray}
	H\leq G\leq A\leq Q
\end{eqnarray}

\section{Polynomes}
\begin{eqnarray}
	P(x)=\sum_{k=0}^na_kx^k
\end{eqnarray}
\emph{Zeros} are $x_i$ that satisfies $P(x_i)=0$.
\subsection{Second degree polynome}
\begin{eqnarray}
	f(x)=ax^2+bx+c&a\neq0
\end{eqnarray}
Can always be rewritten
\begin{eqnarray}
	f(x)=a(x-x_0)(x-x_1)
\end{eqnarray}
where $x_0$ and $x_1$ are the \emph{zeros} of $f(x)$.
Zeros can be determined as follows
\begin{eqnarray}
	x_0,x_1=\frac{-b\pm\sqrt{\Delta}}{2a}\\
	\Delta=b^2-4ac
\end{eqnarray}

For $\Delta$ we have the following properties
\begin{eqnarray}
	\text{if }\Delta>0\text{, then }x_0,x_1\in\mathbb R\\
	\text{if }\Delta=0\text{, then }x_0=x_1\\
	\text{if }\Delta<0\text{, then }x_0,x_1\in\mathbb C
\end{eqnarray}
\begin{myExample}
	Find the zeros of $f(x)=x^2+7x+12$
	
	\begin{eqnarray*}
		\Delta = 7^2-4\cdot1\cdot12=1\\
		x_0=\frac{-7+1}{2}=-3\\
		x_1=\frac{-7-1}{2}=-4
	\end{eqnarray*}
	And $f(x)$ can be rewritten
	\begin{eqnarray*}
		f(x)=(x+4)(x+3)
	\end{eqnarray*}
\end{myExample}

\begin{myExample}
	Find the zeros of $f(x)=2x^2-5x+\frac{25}{4}$
	
	\begin{eqnarray*}
		\Delta = 5^2-4\cdot2\cdot\frac{25}{8}=-25\\
		\sqrt{\Delta}=5i\\
		x_0=\frac{5+5i}{4}=\frac{5}{4}(1+i)\\
		x_1=\frac{5-5i}{4}=\frac{5}{4}(1-i)
	\end{eqnarray*}
	Thus $f(x)$ can be rewritten
	\begin{eqnarray*}
		f(x)=2\left(x-\frac{5}{4}(1+i)\right)\left(x+\frac{5}{4}(i-1)\right)
	\end{eqnarray*}
\end{myExample}

\begin{myExample}
	Find the zeros of $f(x)=x^2+x+1$
	
	\begin{eqnarray*}
		\Delta = 1^2-4\cdot1\cdot1=-3\\
		\sqrt{\Delta}=\sqrt{3}i\\
		x_0=\frac{-1+\sqrt{3}i}{2}\\
		x_1=-\frac{1+\sqrt3i}{2}
	\end{eqnarray*}
	Thus $f(x)$ can be rewritten
	\begin{eqnarray*}
		f(x)=\left(x+\frac{1+\sqrt3i}{2}\right)\left(x+\frac{1-\sqrt{3}i}{2}\right)
	\end{eqnarray*}
\end{myExample}
La relation de Viète nous dit que pour $f(x)=ax^2+bx+c$, les zéros $x_0$ et $x_1$ satisfont
\begin{eqnarray}
	\begin{cases}
		x_0+x_1=\frac{b}{a}\\
		x_0x_1=\frac{c}{a}
	\end{cases}
\end{eqnarray}