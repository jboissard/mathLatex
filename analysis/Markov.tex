\section{Markov Chain}

A markov chain describes the evolution of a system state over time given that we know the change of state probability for each time and state. It is described by 

\begin{equation}
\mathbf{b}_{n+1} = A\mathbf{b}_{n}
\end{equation}

where $\mathbf{b}$ is a $n\times 1$ vector and $A$ a $n\times n$ matrix; $n$ denoting the number of state in the system. The element $a_{ij}$ can be read as the probability to jump from state $i$ to state $j$. Note that $\sum_{j=1}^n a_{ij}=1~\forall i\in(1, .., n)$, $\sum_{i=1}^nb_i=1$ and $a_{ij},b_i\in(0,1)$ 

If $A$ remains constant over time ($A=A(n)$) the state of the system at time $n$ is
\begin{equation}
\mathbf{b}_n = A^{n-k}\mathbf{b}_k= A^n\mathbf{b}_0
\end{equation}

Sometimes there exists a state where the system reaches an equilibrium and is described when

\begin{equation}
\mathbf{b} = A\mathbf{b}
\end{equation}

\subsection{Market evolution}

The market shift from bull to bear and recession can be described (see wikipedia example and PUT IMAGE) using a Markov chain.

If state 
\begin{enumerate}
\item is Bull market
\item is Bear Market
\item is recession
\end{enumerate}
and we have

\begin{equation}
A =
\begin{pmatrix}
0.9	& 0.15	& .25 \\
0.075	& .8		& .25\\
0.025 & 0.05		& .5
\end{pmatrix}
\end{equation}

(e.g. transition from bull to bear market is $a_{21}=7.5\%$). One can show that eventually the market will tend to
\begin{equation}
\mathbf{b} = A\mathbf{b} = 
\begin{pmatrix}
62.5\%\\
31.25\%\\
6.25\%
\end{pmatrix}
\end{equation}