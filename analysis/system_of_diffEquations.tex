\chapter{System of Differential Equations}

\section{Résoudre un système d'équations différentielles linéaire}
Soit
\begin{eqnarray}
	\mathbf{\dot x}=A\mathbf{x}
\end{eqnarray}
où $\mathbf{x}\in\mathbb C^n$ et $A\in \mathbb C^{n\times n}$
ce qui équivaut à
\begin{eqnarray}
	\begin{cases}
		\dot x_1=
		\frac{dx_1}{dt}=
		a_{11}x_1+a_{12}x_2+\dots+a_{1n}x_n
		\\
		\dot x_2=
		\frac{dx_2}{dt}=
		a_{21}x_1+a_{22}x_2+\dots+a_{2n}x_n
		\\
		\vdots 
		\\
		\dot x_n=
		\frac{dx_n}{dt}=
		a_{n1}x_1+a_{n2}x_2+\dots+a_{nn}x_n
	\end{cases}
\end{eqnarray}
ou
\begin{eqnarray}
	\begin{pmatrix}
		\dot x_1
		\\
		\dot x_2
		\\
		\vdots
		\\
		\dot x_n
	\end{pmatrix}
	=
	\begin{pmatrix}
		a_{11} & a_{12} & \cdots & a_{1n}
		\\
		a_{21} & a_{22} & \cdots & a_{2n}
		\\
		\vdots & & &\vdots
		\\
		a_{n1} & a_{n2} & \cdots & a_{nn}
	\end{pmatrix}
	\begin{pmatrix}
		x_1
		\\
		x_2
		\\
		\vdots
		\\
		x_n
	\end{pmatrix}
\end{eqnarray}
La solution est donnée par
\begin{eqnarray}
	\label{eq:generalSolutionLinDiffEq}
	\mathbf{x}(t)=\sum_{i=1}^n c_i \mathbf{v}_i e^{\lambda_it}
\end{eqnarray}
où $\lambda_i\in\mathbb C$ et $\mathbf{v}_i\in\mathbb C^n$ sont respectivement les valeurs propres et vecteurs propres de la matrice $A$. Les $c_i\in\mathbb C$ sont des constantes qui peuvent être déterminée en appliquant des conditions aux systèmes (i.e. conditons initiales).

\subsection*{Exemple\footnote{Example taken from Nonlinear Dynamic and Chaos by Steven H. Strogatz 1994 Perseus book publishing on page 131}}
Solve the initial value problem $\dot x=x+y, \dot y=4x-2y$, subject to the initial condition $(x_0,y_0)=(2, 3)$.
\\

\emph{Solution:}
The corresponding matrix equation is
\begin{eqnarray*}
	\begin{pmatrix}
		\dot x
		\\
		\dot y
	\end{pmatrix}
	=
	\begin{pmatrix}
		1 & 1
		\\
		4 & -2
	\end{pmatrix}
	\begin{pmatrix}
		x 
		\\
		y
	\end{pmatrix}
\end{eqnarray*}
First we find the eigenvalues of the matrix $A$. 
\begin{eqnarray*}
	\det(A-\lambda I)=
	\det\begin{bmatrix}
		1-\lambda & 1
		\\
		4 & -2-\lambda
	\end{bmatrix}
	=
	-(1-\lambda)(2+\lambda)-4=
	\lambda^2-\lambda+6
	=0
\end{eqnarray*}
which gives $\lambda_1=2, \lambda_2=-3$.
\\

Next we find the eigenvectors. Given an eigenvalue $\lambda$, the corresponding eignevector $\mathbf{v}=(v_1,v_2)$ satisfies
\begin{eqnarray}
	\begin{pmatrix}
		1-\lambda & 1
		\\
		4 & -2-\lambda
	\end{pmatrix}
	\begin{pmatrix}
		v_1
		\\
		v_2
	\end{pmatrix}
	=
	\begin{pmatrix}
		0
		\\
		0
	\end{pmatrix},
\end{eqnarray}
which has a nontrivial solution $(v_1,v_2)=(1,1)$, or any scalar multiple thereof.(Of course any multiple of an eigenvector is always an eignenvector; we try to pick the simplest multiple, but any one will do.) Similarly, for $\lambda_2=-3$, the eigenvector equation becomes
\begin{eqnarray*}
	\begin{pmatrix}
		4 & 1
		\\
		4 & 1
	\end{pmatrix}
	\begin{pmatrix}
		v_1
		\\
		v_2
	\end{pmatrix}
	=
	\begin{pmatrix}
		0
		\\
		0
	\end{pmatrix}
\end{eqnarray*}
which has a nontrivial solution $(v_1,v_2)=(1,-4)$. In summary, 
\begin{eqnarray*}
	\mathbf{v_1}=
	\begin{pmatrix}
		1
		\\
		1
	\end{pmatrix}, 
	\mathbf{v_2}=
	\begin{pmatrix}
		1
		\\
		-4
	\end{pmatrix}
\end{eqnarray*}
Next we write the general solution as a linear combination of eigensolutions.
\\

From (\ref{eq:generalSolutionLinDiffEq}), the general solution is
\begin{eqnarray}
	\label{eq:generalSolutionLinDiffEqEx}
	\mathbf{x}(t)=c_1
	\begin{pmatrix}
		1
		\\
		1
	\end{pmatrix}
	e^{2t}
	+
	c_2
	\begin{pmatrix}
		1
		\\
		-4
	\end{pmatrix}
	e^{-3t}
\end{eqnarray}
Finally, we compute $c_1$ and $c_2$ to satisfy the initial condition $(x_0,y_0)=(2,-3)$. At $t=0$, (\ref{eq:generalSolutionLinDiffEqEx}) becomes
\begin{eqnarray*}
	\begin{pmatrix}
		2
		\\
		-3
	\end{pmatrix}
	=
	c_1
	\begin{pmatrix}
		1
		\\
		1
	\end{pmatrix}
	+
	c_2
	\begin{pmatrix}
		1\\-4
	\end{pmatrix},
\end{eqnarray*}
which is equivalent to the algebraic system
\begin{eqnarray*}
	\begin{cases}
		2 = c_1+c_2
		\\
		-3=c_1-4c_2
	\end{cases}
\end{eqnarray*}
The solution is $c_1=1, c_2=1$. Substituting back into (\ref{eq:generalSolutionLinDiffEqEx}) yields
\begin{eqnarray*}
	\begin{cases}
		x(t)=e^{2t}+e^{-3t}
		\\
		y(t)=e^{2t}-4e^{-3t}
	\end{cases}
\end{eqnarray*}
for the solution to the initial value problem.

\section{Nonlinear systems}
	Consider
	\begin{eqnarray}
		\mathbf{\dot x}=\mathbf{f(x)}
	\end{eqnarray}
	where $\mathbf{x}=(x_1,\dots,x_n)\in\mathbb C^n$ and $\mathbf{f(x)}=(f_1(\mathbf{x}), f_2(\mathbf{x}),\dots,f_n(\mathbf{x})): \mathbb C^n\rightarrow\mathbb C^n$.
\\

\subsection{Fixed points}

fixed points satisfy $\mathbf{f(x^*)}=\mathbf{0}$
\subsection{Fixed points and Linearization}
consider the system
\begin{eqnarray}
	\begin{cases}
		\dot x= f(x,y)
		\\
		\dot y= g(x,y)
	\end{cases}
\end{eqnarray}
and suppose that $(x*,y*)$ is a fixed point, i.e.,	$f(x^*,y^*)=0$ and $g(x^*,y^*)=0$.

Let	$u = x - x^*$ and $v= y -y^*$

We have
\begin{eqnarray}
	\dot u = \dot x &\text{(since $x^*$ is a constant)}
	\\
	=f(x^*+u, y^* + v) &\text{(by substitution)}
	\\
	=f(x^*,y^*)
		+u\frac{\partial f}{\partial x}
		+v\frac{\partial f}{\partial y}
		+O(u^2,v^2, uv) &\text{(Taylor expansion)}
		\\
	=u\frac{\partial f}{\partial x}
		+v\frac{\partial f}{\partial y}
		+O(u^2,v^2, uv) &\text{(since $f(x^*, y^*)=0$)}
\end{eqnarray}
Similarly, we find
\begin{eqnarray}
	\dot v=u\frac{\partial g}{\partial x}
		+v\frac{\partial g}{\partial y}
		+O(u^2,v^2, uv).
\end{eqnarray}
Hence the disturbance $(u, v)$ evolves according to
\begin{eqnarray}
	\label{eq:NonLinearSysJaco}
	\begin{pmatrix}
		\dot u
		\\
		\dot v
	\end{pmatrix}
	=
	\begin{pmatrix}
		\frac{\partial f}{\partial x} & \frac{\partial f}{\partial y}
		\\
		\frac{\partial g}{\partial x} & \frac{\partial g}{\partial y}
	\end{pmatrix}
	\begin{pmatrix}
		u
		\\
		v
	\end{pmatrix}
	+
	\text{quadratic terms}.
\end{eqnarray}
The matrix
\begin{eqnarray}
	J=
	\begin{pmatrix}
		\frac{\partial f}{\partial x} & \frac{\partial f}{\partial y}
		\\
		\frac{\partial g}{\partial x} & \frac{\partial g}{\partial y}
	\end{pmatrix}_{(x^*,y^*)}
\end{eqnarray}
is the \textbf{Jacobian} evaluated at the fixed points $(x^*,y^*)$.
\\

Now since the quadratic terms in (\ref{eq:NonLinearSysJaco}) are tiny 
\begin{comment}
	and, as stated before, $\begin{pmatrix}
		\dot x
		\\
		\dot y
	\end{pmatrix}
	=
	\begin{pmatrix}
		\dot u
		\\
		\dot v
	\end{pmatrix}$	
\end{comment}
we obtain the \textbf{\emph{linearized system}}
\begin{eqnarray}
	\begin{pmatrix}
		\dot x
		\\
		\dot y
	\end{pmatrix}
	=
	\begin{pmatrix}
		\frac{\partial f}{\partial x} & \frac{\partial f}{\partial y}
		\\
		\frac{\partial g}{\partial x} & \frac{\partial g}{\partial y}
	\end{pmatrix}_{(x^*,y^*)}
	\begin{pmatrix}
		x
		\\
		y
	\end{pmatrix}
\end{eqnarray}
whose dynamics can be analyzed by the methods of the previous section.
