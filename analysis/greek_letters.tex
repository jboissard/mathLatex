\section{Greek Letters}
\begin{table}[h!]
	\centering
	\begin{tabular}{|c|c|l|}
		\hline 
		\textbf{Lower case}&\textbf{Upper case}&\textbf{Name}\\
		\hline
	    \hline
		$\alpha$ & A & alpha\\ 
		\hline
		$\beta$ & B & beta\\ 
		\hline
		$\gamma$ & $\Gamma$ & gamma\\ 
		\hline
		$\delta$ &$\Delta$ & delta\\ 
		\hline
		$\epsilon$ or $\varepsilon$ & E & epsilon\\ 
		\hline
		$\zeta$ & Z & zeta\\ 
		\hline
		$\eta$ & H & eta\\ 
		\hline
		$\theta$ or $\vartheta$ &$\Theta$ & theta\\ 
		\hline
		$\iota$ &I& iota\\ 
		\hline
		$\kappa$ &K& kappa \\ 
		\hline
		$\lambda$& $\Lambda$ & lambda\\ 
		\hline	
		$\mu$ &M& mu \\ 
		\hline
		$\nu$ &N& nu \\ 
		\hline
		$\xi$ &$\Xi$ & xi \\ 
		\hline
		$o$ & O& omicron \\ 
		\hline
		$\pi$ or $\varpi$&$\Pi$ & pi\\ 
		\hline
		$\rho$ ou $\varrho$ &P& rho\\ 
		\hline
		$\sigma$ or $\varsigma$& $\Sigma$ & sigma\\ 
		\hline
		$\tau$ &T& tau\\ 
		\hline
		$\upsilon$& $\Upsilon$ & upsilon\\ 
		\hline
		$\phi$ or $\varphi$ &$\Phi$& phi\\ 
		\hline
		$\chi$ &X& khi\\ 
		\hline
		$\psi$ &$\psi$& psi\\ 
		\hline
		$\omega$ & $\Omega$ & omega\\ 
		\hline
	 \end{tabular}	
	\caption{Greek letters}
	\label{table:greek_letters}
\end{table}

\section{Famous constants}
\subsection{$\pi$ number}
\begin{eqnarray}
	\pi = 3.14159\,26535\,89793\,23846\,26433\ldots
\end{eqnarray}
\subsection{Euler's number $e$}
\begin{eqnarray}
	e=\lim_{n\rightarrow+\infty}\left(1+\frac{1}{n}\right)^n\\
	e=\sum_{n=0}^{\infty}\frac{1}{n!}\\
	e=2.71828\,18284\,59045\,23536\ldots\,
\end{eqnarray}

\subsection{Le nombre d'or / Golden Ratio}
\begin{eqnarray}
	\label{eq:goldenratio}
	\Phi=\frac{\sqrt5 + 1}{2}\\
	\frac{1}{\Phi}=\Phi - 1\\
	\Phi \approx 1.61803\,39887\ldots\,
\end{eqnarray}

\subsection{Fibonnacci numbers}
In mathematics, the Fibonacci numbers are the numbers in the following sequence:
\begin{eqnarray}
	0, 1, 1, 2, 3, 5, 8, 13, 21,\dots
\end{eqnarray}
By definition, the first two Fibonacci numbers are 0 and 1, and each remaining number is the sum of the previous two. Some sources omit the initial 0, instead beginning the sequence with two 1s.

Which can be rewritten
\begin{eqnarray}
	\begin{cases}
		u_{n}=u_{n-1}+u_{n-2} & n\geq2\\
		u_0=0\\
		u_1=1
	\end{cases}
\end{eqnarray}

\subsubsection{Relation to the Golden Ratio}
\begin{eqnarray}
	u_n=\frac{\varphi^n-(1-\varphi)^n}{\sqrt{5}}=\frac{\varphi^n-(-1/\varphi)^n}{\sqrt{5}}=\frac{(1+\sqrt{5})^n-(1-\sqrt{5})^n}{2^n\sqrt{5}}
\end{eqnarray}
where $\varphi$ is the golden ratio (Equation: \ref{eq:goldenratio}).

\chapter{Conversion unité}
\section{Fahrenheit - Celsius}
\begin{eqnarray}
	T_C=\frac{5}{9}(T_F-32)\\
	T_F=\frac{9}{5}T_C+32
\end{eqnarray}
Thus $-40\celsius=-40\degree F$.

\section{Notations}
Here below are listed some common abbreviations used in mathematics, physics or economics.


\begin{table}[h!]
	\centering
	\begin{tabular}{|c|c|}
		\hline 
		\textbf{Abbreviation}&\textbf{Meaning}\\
		\hline
	    \hline
		$\dot x$ &  $\frac{dx}{dt}$\\ 
		\hline
		$(f)'$ &  $\frac{df}{dx}$\\ 
		\hline
		$f_x$ &  $\frac{\partial x}{\partial t}$\\ 
		\hline
	 \end{tabular}	
	\caption{Abbreviations}
	\label{table:abbreviations}
\end{table}

