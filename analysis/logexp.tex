%!TEX root = /Users/Johan/Documents/University/Math/Gymnase/book.tex
\chapter{Fonctions logarithmiques et exponentielles}
\section{Fonctions réciproques}

\begin{myDefinition}
	\textbf{Injective} $f(x)$ et injective quand $f(a)\neq f(b)\Leftrightarrow a\neq b$, c-à-d la fonction n'a jamais deux fois la même valeur, elle est donc soit strictement croissante ou décroissante.
\end{myDefinition}

\begin{myDefinition}
	\textbf{Réciproque} $y=f(x) \Leftrightarrow x=g(y)$, alors $g(y)$ est la fonction réciproque de $f(x)$.

On note la réciproque de $f$, $f^{-1}(x)$.

On a également la propriété suivante: $f(f^{-1}(x))=f^{-1}(f(x))=x$

Si $g(x)=f^{-1}(x)$ alors $g'(x)=\frac{1}{f'(g(x))}$
\end{myDefinition}

\begin{myExample}
	$f(x)=3x-5$, trouver la fonction réciproque.\\\\
	On pose tout d'abord $y=f(x)$ et donc $y=3x-5$. Par raisonnement algébrique on trouve $x=\frac{y+5}{3}$ et donc $f^{-1}(x)=\frac{x+5}{3}$
\end{myExample}
\section{Logarithme naturel}
\begin{myDefinition}
	\begin{eqnarray}
		\ln{x}=\int_1^x\frac{1}{t}dt
	\end{eqnarray}
	On en déduit tout de suite que
	\begin{eqnarray}
		\frac{d}{dx}\ln{x}=\frac{1}{x}
	\end{eqnarray}
\end{myDefinition}

On a les propriétés suivantes:
\begin{eqnarray}
	\ln{(ab)}=ln{a}+\ln{b}
	\\
	\ln{\frac{a}{b}}=\ln{a}-\ln{b}
	\\
	\ln{a^b}=b\ln{a}
\end{eqnarray}
\subsection{Dérivation logarithmique}
\begin{eqnarray*}
	y=f(x)
	\\
	\ln y=\ln f(x)
	\\
	\frac{1}{y}y'=\frac{f'(x)}{f(x)}=(\ln{f(x)})'
\end{eqnarray*}
\begin{myExample}
	\begin{eqnarray*}
		y=\frac{(5x-4)^3}{\sqrt{2x-1}}
		\\
		\ln{y}=\ln{\left(\frac{(5x-4)^3}{\sqrt{2x-1}}\right)}
		\\
		=\ln{\left((5x-4)^3\right)}-\ln{\left(\sqrt{2x-1}\right)}
		\\
		=3\ln{(5x-4)}-\frac{1}{2}\ln{(2x-1)}
	\end{eqnarray*}
	Maintenant il est facile d'obtenir une dérivée.
	\begin{eqnarray*}
		(\ln{y})'=3\frac{5}{5x-4}-\frac{1}{2}\frac{2}{2x-1}=\frac{15}{5x-4}-\frac{1}{2x-1}
	\end{eqnarray*}
\end{myExample}
\section{Fonction exponentielle}
\begin{myDefinition}
	La fonction exponentielle est la fonction réciproque du logarithme.
	\begin{eqnarray}
		e^x=\exp{x}=y \Leftrightarrow x=\ln y
	\end{eqnarray}
	On a donc 
	\begin{eqnarray}
		\exp{(\ln{y})}=\ln{(e^y)}=y
	\end{eqnarray}
	Notons que $e^1=e\backsimeq 2.71828\dots$
\end{myDefinition}
On a les relations sivantes
\begin{eqnarray}
	e^pe^q=e^{p+q}
	\\(e^{p})^r=e^{pr}
	\\
	\label{eq:expdiff}(e^x)'=e^x
	\\
	(e^u)^r=u'e^u
	\end{eqnarray}
Notez l'équation \ref{eq:expdiff} qui est particulièrement intéressante. L'exponentielle est la seule fonction à posséder cette propriété; $f(x)=f'(x)$.
\section{intégration}
\begin{eqnarray}
	\int\frac{1}{u}du=\ln |u|+C
	\\
	\int e^udu=\frac{e^u}{u'}+C
	\\
	\int\tan{(u)}du=-\ln{|\cos{u}|}+C
	\\
	\int\cot{u}du=\ln{|\cos{u}|}+C
	\\
	\int\sec{u}du=\ln{|\sec{u}+\tan{u}|}+C
	\\
	\int\csc{u}du=\ln{|\csc{u}+\cot{u}|}+C
\end{eqnarray}
\textbf{Rappel:}
\begin{eqnarray}
	\tan^{-1}{u}=\cot{u}
	\\
	\sin^{-1}{u}=\csc{u}
	\\
	\cos^{-1}{u}=\sec{u}
\end{eqnarray}

\section{Fonctions exponentielles et logarithmes de bases quelconque}
\begin{eqnarray}
	a^x=e^{x\ln{(a)}}
	\\
	(a^x)'=a^x\ln{a}
	\\
	(a^u)'=a^u\ln{(a)}u'
	\\
	\int a^xdx=\frac{a^x}{\ln{(a)}}
	\\
	\int a^udx=\frac{a^u}{\ln{(a)}u'}
\end{eqnarray}

\begin{eqnarray}
	y=\log_ax\Leftrightarrow x=a^y
	\\
	\log_ax=\frac{\ln x}{\ln a}
\end{eqnarray}

\begin{eqnarray}
	(\log_ax)'=\left(\frac{ln x}{\ln a}\right)'=\frac{1}{\ln(a)}\frac{1}{x}
\end{eqnarray}

Pour prouver que $\log_ax=\frac{\ln{x}}{\ln{a}}$, on part de
\begin{eqnarray*}
	a^y=x
\end{eqnarray*}
Par définition, on a
\begin{eqnarray*}
	\log_a{a^y}=y=\log_a x
\end{eqnarray*}
Mais aussi
\begin{eqnarray*}
	\ln{a^y}=y\ln{a}=\ln{x}
\end{eqnarray*}
et donc
\begin{eqnarray*}
	y=\frac{\ln{x}}{\ln{a}}=\log_ax
\end{eqnarray*}
Notons que ce résultat se généralise aisément pour obtenir
\begin{eqnarray}
	\log_a{x}=\frac{\log_bx}{\log_ba}
\end{eqnarray}
On peut vérfier le théorème précédent de la façon suivante
\begin{eqnarray*}
	\ln{(x)}=\log_a{(x)}\ln{(a)}=\ln{(a^{\log_a{(x)}})}\\
	x=a^{\log_a{(x)}}=x
\end{eqnarray*}

\section{Loi de croissance}
\begin{eqnarray}
	y(t)=y_0e^{kt}
\end{eqnarray}
où $y_0$ est la valeur initiale, à $t=0$. $y_0$ a la même unité que $y$.
$k$ est une constante d'unité $[\frac{1}{s}]$, si elle est multipliée au temps $t$. L'exposant, ici $kt$, n'a jamais d'unité.

\section{Factorial}
\begin{equation}
	n! = \prod_{k=1}^n k  = n\cdot(n-1)\cdot(n-2)...\cdot1
\end{equation}
where $k\in\mathbb N$

\subsection{Stirling's Approximation}
for large values of $n$, $n!$ can be approximated as following
\begin{equation}
	n! \approx \sqrt{2\pi n}\left(\frac{n}{e}\right)^n.
\end{equation}

\subsection{Binomial Coefficients}
\begin{equation}
	\binom nk = \frac{n!}{(n-k)!k!}
\end{equation}
