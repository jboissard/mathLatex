\chapter{Trigonométrie}
\begin{eqnarray}
	\sin x = y \Leftrightarrow \arcsin y=x & -1\leq x\leq 1 & \frac{\pi}{2}\leq y\leq\frac{\pi}{2}\\
	\cos x = y \Leftrightarrow \arccos y=x & -1\leq x\leq 1 & 0\leq y\leq\pi\\
	\tan x = y \Leftrightarrow \arctan y=x &x\in\mathbb R&-\frac{\pi}{2}\leq y\leq \frac{\pi}{2}
\end{eqnarray}
\section{Valeurs exacts des fonctions trigonométriques d'arcs particuliers}
\begin{tabular}{| c|c|c|c|}
    \hline
	$\alpha$ & $\cos\alpha$ & $\sin\alpha$ & $\tan\alpha$\\
    \hline
    $0\textdegree$  0 & 1 & 0 & 0\\
    \hline
	$30\textdegree$ $\frac{\pi}{6}$ & $\frac{\sqrt{3}}{2}$ & $\frac{1}{2}$ & $\frac{\sqrt 3}{3}$\\
	\hline
	$45\textdegree$ $\frac{\pi}{4}$ & $\frac{\sqrt{2}}{2}$ & $\frac{\sqrt 2}{2}$ & $1$\\
	\hline
	$60\textdegree$ $\frac{\pi}{3}$  & $\frac{1}{2}$ & $\frac{\sqrt{3}}{2}$ & $\sqrt 3$\\
	\hline
	$90\textdegree$ $\frac{\pi}{2}$  & $0$ & $1$ & -\\
	\hline
 \end{tabular}
\section{Périodicité des fonctions trigonométriques}
\begin{eqnarray}
	\cos{(\alpha+2\pi)}=\cos{(\alpha)}\\
	\sin{(\alpha+2\pi)}=\sin{(\alpha)}\\
	\tan{(\alpha+\pi)}=\tan{(\alpha)}
\end{eqnarray}
\section{Relations entre fonctions trigononmétriques d'un même arc}
\begin{tabular}{|l|l|l|}
    \hline
	$\cos^2{(\alpha)}+\sin^2{(\alpha)}=1$ &
	$\tan{(\alpha)}=\frac{\sin{(\alpha)}}{\cos{(\alpha)}}$ &
	$\cot{(\alpha)}=\frac{\cos{(\alpha)}}{\sin{(\alpha)}}$ \\
	$\cot{(\alpha)}=\frac{1}{\tan{(\alpha)}}$ &
	$\frac{1}{\cos^2{(\alpha)}}=1+\tan^2{(\alpha)}$ &
	$\frac{1}{\sin^2{(\alpha)}}=1+\cot^2{(\alpha)}$  \\
	\hline
\end{tabular}
\section{Relations entre fonctions trigononmétriques de certains arcs}
\begin{tabular}{|l|l|l|}
    \hline
	$\cos{(-\alpha)}=\cos{(\alpha)}$ & $\sin{(-\alpha)}=-\sin{(\alpha)}$ & $\tan{(-\alpha)}=-\tan{(\alpha)}$ \\
   % \hline
	$\cos{(\pi-\alpha)}=-\cos{(\alpha)}$ & $\sin{(\pi-\alpha)}=\sin{(\alpha)}$ & $\tan{(\pi-\alpha)}=-\tan{(\alpha)}$ \\
    %\hline
	$\cos{(\pi+\alpha)}=-\cos{(\alpha)}$ & $\sin{(\pi+\alpha)}=-\sin{(\alpha)}$ & $\tan{(\pi+\alpha)}=\tan{(\alpha)}$ \\
    %\hline
	$\cos{\left(\frac{\pi}{2}-\alpha\right)}=\sin{(\alpha)}$ & $\sin{\left(\frac{\pi}{2}-\alpha\right)}=\cos{(\alpha)}$ &
	$\tan{\left(\frac{\pi}{2}-\alpha\right)}=\cot{(\alpha)}$ \\
    %\hline
	$\cos{\left(\frac{\pi}{2}+\alpha\right)}=-\sin{(\alpha)}$ & $\sin{\left(\frac{\pi}{2}+\alpha\right)}=\cos{(\alpha)}$ &
	$\tan{\left(\frac{\pi}{2}+\alpha\right)}=-\cot{(\alpha)}$ \\
    \hline

 \end{tabular}
\\\\
\underline{Remarques:} ces relations peuvent être facilement retrouvées en utilisant la formule de Moivre: $e^{j\phi}=\cos{\phi}+j\sin{\phi}$, voir le chapitre sur les nombres complexes.
\section{Conversion des mesures d'angles}
On note respectivement $d,r, m$ et $g$ la mesure d'angle en degrés, en radians, en minutes et en grades.

Pour un même angle, on a 
\begin{eqnarray}
	\frac{d}{180}=\frac{r}{\pi}=\frac{m}{30}=\frac{g}{200}
\end{eqnarray}