\chapter{Les formes indéterminées $\frac{0}{0}$ et $\frac{\infty}{\infty}$}
\section{La règle de l'Hôpital}
\begin{myTheorem}
	Si\begin{eqnarray*}
		\lim_{x \to c}f(x)=\lim_{x \to c}g(x)=0\text{ ou }\infty 
	\end{eqnarray*}
	et si $\lim_{x\to c}f'(x)/g'(x)$ existe, alors
	\begin{eqnarray}
		\lim_{x\to c}\frac{f(x)}{g(x)} = \lim_{x\to c}\frac{f'(x)}{g'(x)}.
	\end{eqnarray}
\end{myTheorem}

\begin{myExample}
	On veut calculer
	\begin{eqnarray*}
		\lim_{x\rightarrow0}\frac{\cos{(x)}+2x-1}{3x}
	\end{eqnarray*}
	On remarque qu'on tombe sur une forme indéterminée ($\frac{0}{0}$). En dérivant l'expression du haut et du bas on trouve
	\begin{eqnarray}
		\lim_{x\rightarrow0}\frac{\cos{(x)}+2x-1}{3x}=\lim_{x\rightarrow0}\frac{\sin{(x)}+2}{3}=\frac{2}{3}
	\end{eqnarray}
\end{myExample}