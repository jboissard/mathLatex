%
%  untitled
%
%  Created by Johan Boissard [] on 2010-06-24.
%  Copyright (c) Johan Boissard. All rights reserved.
% hhh

\documentclass[a4paper] {scrartcl}
\usepackage[T1]{fontenc}
\usepackage[utf8]{inputenc}
\usepackage{graphicx}
\usepackage{engord}
%\usepackage[english]{babel}
\usepackage{fancyhdr}
\usepackage{amsmath, amssymb}
\usepackage{comment}

\usepackage{listings}

%allows inclusion of url (hyperref is better than url) 
%ref: http://www.fauskes.net/nb/latextips/
\usepackage{hyperref}

%package for chemistry ie: \ce{(NH4)2SO4 -> NH4+ + 2SO4^2-} 
%ref:www.ctan.org/tex-archive/macros/latex/contrib/mhchem/mhchem.pdf
\usepackage[version=3]{mhchem}
%celsius + degrees
\usepackage{gensymb}
%to get last page
\usepackage{lastpage} % \pageref{LastPage}

%make use of the fullpage (no HUGE margins)
\usepackage{fullpage}
\usepackage{subfig}

%allows separating cell in table by diagonal line
\usepackage{slashbox}




%\renewcommand{\chaptername}{Laboratory}
%\setcounter{chapter}{5}

\usepackage{color}
\usepackage[usenames,dvipsnames, table]{xcolor}
% Include this somewhere in your document



\usepackage[absolute]{textpos}

%column  of multi row in tables
\usepackage{multirow}

%to have vertical text in table
\usepackage{rotating}


%%%%%%% a virer ici!!!!
\begin{comment}
%Fonts and Tweaks for XeLaTeX
\usepackage{fontspec,xltxtra,xunicode}
%\defaultfontfeatures{Mapping=tex-text}
%\setromanfont[Mapping=tex-text]{Hoefler Text}
\setsansfont[Scale=MatchLowercase,Mapping=tex-text]{Gill Sans}

\definecolor{shade}{HTML}{D4D7FE}	%light blue shade
\definecolor{text1}{HTML}{272727}		%text is almost black
\definecolor{headings}{HTML}{173849} 	%dark blue %%%dark red 70111
\definecolor{title}{HTML}{173849} 	%dark blue %%%dark red 70111

\usepackage{titlesec}				%custom \section
\end{comment}







\author{Johan Boissard}
\date{\today}
\title{Taylor Series}
\begin {document}

\maketitle
%\tableofcontents

\section{Théorie}
La série de Taylor d'une fonction $f(x)$ qui est infiniment différentiable dans le voisinage de $\alpha$ est

\begin{equation}
	f(x)\approx \sum_{k=0}^n f^{(k)}(\alpha)\frac{(x-\alpha)^k}{k!}
\end{equation}


\section{Exemple}
\subsection{Série de Taylor d'une fonction polynomiale autour de $\alpha=0$}
\begin{eqnarray*}
	f(x) = x^a\\
	f^{(k)}(x) = \frac{a!}{(a-k)!}x^{a-k}\\
	f^{(k)}(\alpha)=f^{(k)}(0) = 
	\begin{cases}
			0 \text{  }a\neq k\\
			a! \text{  }a=k
	\end{cases}
\end{eqnarray*}

On voit donc que la série se résume à 
\begin{equation}
	f(x) = a!\frac{(x-0)^a}{a!}=x^a
\end{equation}

\subsection{Trouver la valeur de $\ln{(2)}$}
Pour cela, on cherche la série de taylor de $\ln{(x)}$ autour de $\alpha=1$.

\begin{eqnarray}
	f(x) = \ln{(x)}\\
	f(1) = 0\\
	f^{(k)}(x) = \frac{(-1)^{(k+1)}(k-1)!}{x^k}\\
	f^{(k)}(1) = (-1)^{(k+1)}(k-1)!
\end{eqnarray}

On a donc ($\alpha=1$)
\begin{equation}
	\ln{(x)} \approx \sum_{k=1}^n (-1)^{(k+1)}\frac{(x-1)^k}{k}
\end{equation}
Si on se limite à $n=4$ on obtient

\begin{equation}
	\ln{(x)} \approx (x-1) -\frac{(x-1)^2}{2} + \frac{(x-1)^3}{3} - \frac{(x-1)^4}{4}
\end{equation}
et donc pour $\ln{(2)}$, on obtient $\ln{(2)} \approx 1 - \frac{1}{2}+\frac{1}{3}-\frac{1}{4}=.583333$ qui n'est pas trop éloigné de $\ln{(2)}=.69$. Plus $n$ grandit et plus l'écart entre la vraie valeur et la valeur calculée diminue.



Il est facile de montrer que (par extension de ce qui a été fait précédemment) que
\begin{equation}
	\ln{(2)} = \sum_{k=1}^\infty \frac{(-1)^{(k+1)}}{k}
\end{equation}





\end{document}
