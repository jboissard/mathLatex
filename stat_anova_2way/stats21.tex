\documentclass[a4paper]{article}

% Packages
\usepackage[french]{babel}
\usepackage[utf8]{inputenc}
\usepackage[T1]{fontenc}
\usepackage{amsmath, tikz, fullpage}

\usepackage{subfigure}

% Title
\title{Homework 01}
\author{Johan Boissard}
\date\today

% Header
\usepackage{fancyhdr}
\setlength{\headheight}{15.2pt}
\setlength{\headsep}{15pt}
\pagestyle{fancy}

\fancyhf{}
\lhead{Probabilités et statistiques}
\chead{Série 21}
\rhead{Johan Boissard}


\begin{document}
\section*{Exercice 2}
\subsection*{a.)}

Modèle ANOVA à deux voies avec intéractions
\begin{eqnarray*}
	Y_{ijk}=\mu+\alpha_i+\beta_j+\gamma_{k}+\epsilon_{ijk}
\end{eqnarray*}

D'après
\begin{eqnarray*}
	\text{Total corrigé:} Y_{\dots}=\sum_{i=1}\sum_{j=1}\sum_{k=1}(y_{ijk}-y_{...})
	\\
	\hat{\alpha}_i=Y_{i..}-Y_{...}
	\\
	\hat{\beta}_j=Y_{.j.}-Y{...}
	\\
	\hat\gamma=Y{ij.}-\hat\alpha_i-\hat\beta_j
	\\
	r_{ijk}=Y_{ijk}-\hat\mu-\hat\alpha_i-\hat\beta_j-\hat\gamma_{ij}=Y{ijk}-\hat Y{ijk}
\end{eqnarray*}
On trouve le tableau suivant: \ref{table:2}

% previous version disgusting !
%\includegraphics[height=130mm]{la.jpeg}
	
% \begin{tabular}{l|r|r|r|r|r|}
% 		& 1		& 2		& 3		& 4		& 5		\\
% 	\hline
% 	A	& 33	& 31	& 34	& 34	& 31	\\
% 		& 36	& 31	& 36	& 33	& 31	\\
% 	\hline
% 	B	& 32	& 37	& 40	& 33	& 35	\\
% 		& 35	& 35	& 36	& 36	& 36	\\
% 	\hline
% 	C	& 37	& 35	& 34	& 31	& 37	\\
% 		& 39	& 35	& 37	& 35	& 40	\\
% 	\hline
% 	D	& 29	& 31 	& 33	& 31	& 33	\\
% 		& 31	& 33	& 34	& 27	& 33	\\
% 	\hline
% 		& 		& 		& 		& 		& 		\\
% 	\hline
% 		& 34	& 34	& 36	& 33	& 35	\\
% 	\hline
% 		& 		& 		& 		& 		& 		\\
% 	\hline
% 	$\beta$ & 0	& -1	& 1.5	& -2	& .5	\\
% 	\hline
% 		& 		& 		& 		& 		& 		\\
% 	\hline
% $\gamma_{ij}$ & 1.5 & -2 & .5	& 2		& -3	\\
% 	\hline
% 		& 		& 		& 		& 		& 		\\
% 	\hline
% 	 	&  -2	& 1		& 1		& .5	& -1 	\\
% 	\hline
% 		& 		& 		& 		& 		& 		\\
% 	\hline
% 		& 2		& -1 	& -2	& -2	& 2		\\
% 	\hline
% 		& 		& 		& 		& 		& 		\\
% 	\hline
% 		& -1.5	& 1 	& .5	& -1	& 1		\\
% 	\hline
% 		& 		& 		& 		& 		& 		\\
% 	\hline
% Résidus$_{ijk}$ & -1.5 & 0 & -1	& .5	& 0	\\
% 	\hline
% 		& 1.5	& 0		& 1		& -1	& 0 		\\
% 	\hline
% 	 	&  -1.5	& 1		& 2		& -2	& -1 	\\
% 	\hline
% 		& 1.5 	& -1 	& -2	& 1.5	& .5		\\
% 	\hline
% 		& 	-1	& 0		& -2	& -2	& -2		\\
% 	\hline
% 		& 1		& 0 	& 1.5	& 2		& 1.5		\\
% 	\hline
% 		& 	-1	& -1	& -1	& 	2	& 0		\\
% 	\hline
% 		& 1		& 1 	& .5	& -2	& 0		\\
% \end{tabular}
	
\begin{table}[tbp]
\centering
\caption{Table}
\subtable[Problem]{
\begin{tabular}{l|r|r|r|r|r||r|r|}
	$Y_{ijk}=$	& 1		& 2		& 3		& 4		& 5 & &	$\alpha$	\\
	\hline
	A	& 33	& 31	& 34	& 34	& 31 & &	\\
		& 36	& 31	& 36	& 33	& 31 &	33 & -1	\\
	\hline
	B	& 32	& 37	& 40	& 33	& 35 & &	\\
		& 35	& 35	& 36	& 36	& 36 & 35.5 & 1.5	\\
	\hline
	C	& 37	& 35	& 34	& 31	& 37 & &	\\
		& 39	& 35	& 37	& 35	& 40 & 36 &	2\\
	\hline
	D	& 29	& 31 	& 33	& 31	& 33 & &	\\
		& 31	& 33	& 34	& 27	& 33 & 31.5 & -2.5	\\
	\hline
			\\
	\hline
		& 34	& 34	& 36	& 33	& 35 & 34	\\
	\hline
		\\
	\hline
	$\beta$ & 0	& -1	& 1.5	& -2	& .5 	\\
	\hline
\end{tabular}
\label{h5ft5n}
}

\subtable[$\gamma$]{
\begin{tabular}{l|r|r|r|r|r|}
	$\gamma_{ij}$ &		& & & 	& \\
		\hline
		 	& 1.5 & -2 & .5	& 2		& -3	\\
		\hline
			& 		& 		& 		& 		& 		\\
		\hline
		 	&  -2	& 1		& 1		& .5	& -1 	\\
		\hline
			& 		& 		& 		& 		& 		\\
		\hline
			& 2		& -1 	& -2	& -2	& 2		\\
		\hline
			& 		& 		& 		& 		& 		\\
		\hline
			& -1.5	& 1 	& .5	& -1	& 1		\\
		\hline
			& 		& 		& 		& 		& 		\\
		\hline
\end{tabular}}

\subtable[Residues]{
\begin{tabular}{l|r|r|r|r|r|}
	$r_{ijk}$ &		& & & 	& \\
		\hline
	 		& -1.5 & 0 & -1	& .5	& 0	\\
		\hline
			& 1.5	& 0		& 1		& -1	& 0 		\\
		\hline
		 	&  -1.5	& 1		& 2		& -2	& -1 	\\
		\hline
			& 1.5 	& -1 	& -2	& 1.5	& .5		\\
		\hline
			& 	-1	& 0		& -2	& -2	& -2		\\
		\hline
			& 1		& 0 	& 1.5	& 2		& 1.5		\\
		\hline
			& 	-1	& -1	& -1	& 	2	& 0		\\
		\hline
			& 1		& 1 	& .5	& -2	& 0		\\
\end{tabular}

}
\label{table:2}
\end{table}
	
\subsection*{c.)}
See table \ref{table:1}
	
%% old version - with a jpg file -> disgusting	
%\includegraphics[height=40mm]{la2.jpg}
		
\begin{table}[h!]
	\center
	\caption{Anova}
	\begin{tabular}{lrrrr}
		\textbf{Source} &	\textbf{SC} &	\textbf{dl} &	\textbf{CM} &	\textbf{F}	\\
		Secrétaire 		& 40			& 4				& 10			& 3.45			\\
		Machines		& 135			& 3				& 45			& 15.52			\\
		Inter			& 83			& 12			& 6.92			& 2.39			\\
		Erreur			& 58			& 20			& 2.9 			&				\\	
		\hline
		Total			& 316			& 39			&				&				\\
	\end{tabular}
	\label{table:1}
\end{table}
		
		
	\subsection*{d.)}
	On teste l'hypothèse suivante: $H_0$: l'effet du facteur 1 (secrétaires)$=0$. Les degrés de liberté sont: $I-1=5-1=4$ et $IJ(K-1)=4*5*(2-1)=20$. On a $qF_{4,20}(0,95)=2.866$. Donc
	\begin{eqnarray*}
		F_{obs}=3.45>qF_{4,20}(0,95)=2.866
	\end{eqnarray*}
	L'hypothèse est donc rejetée
	\\
	On teste l'hypothèse suivante: $H_0$: l'effet du facteur 2 (machines)$=0$. Les degrés de liberté sont: $J-1=4-1=3$ et $IJ(K-1)=4*5*(2-1)=20$. On a $qF_{3,20}(0,95)=3.098$. Donc
	\begin{eqnarray*}
		F_{obs}=15.52>qF_{3,20}(0,95)=3.098
	\end{eqnarray*}
	L'hypothèse est donc également rejetée.
	\\
	On teste maintenant l'interaction. $H_0$: l'effet de l'intercation (secrétaire/machine)$=0$. Les degrés de liberté sont: $(I-1)(J-1)=(5-1)*(4-1)=12$ et $IJ(K-1)=4*5*(2-1)=20$. On a $qF_{12,20}(0,95)=2.278$. Donc
	\begin{eqnarray*}
		F_{obs}=2.39>qF_{12,20}(0,95)=2.278
	\end{eqnarray*}
	L'hypothèse est également rejetée.
\end{document}